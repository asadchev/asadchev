\documentclass[12pt]{book} \usepackage[margin=1in]{geometry}

\usepackage{graphicx}
\usepackage{amsmath}
\usepackage{color}
\usepackage{setspace}

\newcommand{\bra}[1]{\langle #1|}
\newcommand{\ket}[1]{|#1\rangle}
\newcommand{\braket}[2]{\langle #1|#2\rangle}


%% \usepackage{amsmath}
%% \usepackage{listings}

%% \lstloadlanguages{C++} \lstset{ language=C++, breaklines=true,
%%   keywordstyle=\color{blue}, commentstyle=\color{red} }




\newenvironment{listing}%
               {\begin{table}
                   \begin{tabular}{ p{6in} }
                     \hline}%
               {\end{tabular}%
               \end{table}}


\begin{document}
\doublespacing

\title{Thesis: implementations of core computational chemistry
algorithms} \author{Andrey Asadchev} \date{}

\maketitle

\tableofcontents


\chapter{Introduction} The primary goal of computational chemistry is
of course to predict chemical properties: energy, gradients, Hessians
(vibrational frequencies), and other properties for a given chemical
system.  For example, to find the excitation energy or rotation
barriers one would perform a series of single point energy
calculations.  To find local extrema on the potential energy
surface a series of energy gradient calculations are needed, and so
on.

The computational science aspect of computational chemistry is often
treated as a necessary evil.  Over the years, it so happened that most
of the designers and authors of quantum chemistry algorithms and their
implementations were chemists and physicists first, and computational
scientists second.

The foundations for the quantum chemistry were developed before the
World War II. For example the Hartree-Fock \cite{hartree,fock} method
was developed in the late 20's, and the foundation of perturbation
theory \cite{moller1934note} dates back to the mid-30's.  However,
practical application of the theoretical methods did not come until
the emergence of sufficient computing resources to crunch the numbers.

20th century scientific computing was dominated by Fortran, short for
Formula Translator, one of the earliest programming languages, first
developed in the 50's \cite {fortran}.  The computers and operating
systems at the inception of Fortran were expensive proprietary
products, batch machines running stacks of manually prepared inputs.
Compared to today's powerful computers, computing in the 50's and the
60's may as well have been done on clay tablets.

In the 70's another language, C \cite{c}, and a new operating system,
UNIX, came out of Bell Labs.  With the rise of UNIX, the C programming
language gained strong footing among computer science and computer
engineering practitioners.  In the same decade Cray produced its first
groundbreaking supercomputer, Cray I, which gave researchers for the
first ability to crack tough numerical problems, such as weather
prediction, in a timely manner.  In the field of computational
chemistry many of the core programs (some still in use today) were
developed and incorporated into computational chemistry packages,
notably HONDO \cite{hondo} and GAUSSIAN \cite{gaussian}.  A majority
of this work was spearheaded by John Pople, who won the 1998 Nobel
Prize for his contribution to the field.

The 80's saw the growth of UNIX and standardization of system
interfaces with POSIX and SystemV \cite{isaak1990standards}
standards.  C++ \cite{c++}, a multi-paradigm language based on C, was
being developed by Bjarne Stroustrup at Bell Labs.  To avoid
limitations placed on software by patents and restrictive licensing,
Richard Stallman began the GNU foundation, which sought to liberate
software development.  The GNU Compilers Collection (GCC) and GNU
public licenses are perhaps the most visible of the many contributions
GNU made to computing and scientific fields.  The decade also
witnessed the birth of massively parallel supercomputers, such as the
Thinking Machines.  To take advantage of the emerging trends in
scientific computing, a number of parallel computational chemistry
algorithms were developed, including parallel Hartree-Fock
\cite{dupuis1987parallel} and MP2 \cite{watts1988parallel}.  In the
early 80's Purvis and Bartlett first implemented a coupled cluster
singles and doubles algorithm \cite{purvis1982full}, or CCSD for
short.  Subsequently, CCSD with perturbative triples correction method
\cite{raghavachari1989fifth}, CCSD(T), was developed which today is
the gold standard of computational chemistry.  In the same decade,
GAMESS \cite{gamess} began to be developed, with HONDO as much of its
initial codebase.

In the 90's, the exotic supercomputers of the previous decades slowly
disappeared, starved from generous military budgets of the Cold War
which was now over \cite{nyt92,nyt94}.  The burgeoning personal computer market funneled
billions of dollars into research and development of commodity Intel
and AMD processors.  The fragmented UNIX market was slowly eroded by
the ever maturing Microsoft Windows and a new operating system, Linux.
Started as a hobby in the early 90's by Linus Torvalds, Linux,
released under GNU Public License, quickly caught the interest of
programmers worldwide and within a few years became one the major
operating systems of the Internet age.  The C++ programming language
became the preferred choice for writing complex applications, albeit
not just yet in the scientific fields.  However, more and more
scientific codes of the 90's were run and developed for clusters of
commodity computers running Linux and connected by relatively
inexpensive networks.  One of the more interesting developments in
computational chemistry was NWChem \cite{nwchem}, a set of codes
designed specifically with parallel distributed memory systems in
mind.  NWChem was perhaps the last major computational chemistry
package whose development started primarily in Fortran.

The Internet bubble burst at the turn of the 21st century, spelling
financial problems and consequent death to the many flagship companies
of the last century, including SUN and SGI.  With the release of
X86-64 extensions by AMD in 2003, commodity processors became a
full-fledged 64-bit architecture, suitable for any computational
challenge.  By the 2010's the processor market became dominated almost
exclusively by multicore AMD and Intel chips, with IBM still retaining
some presence in the high-end computing market with its Power
processors.  The latest development in the commodity computing is the
reemergence of accelerators, such as using graphics cards to solve
general programs, so-called General Processing on GPU (GPGPU).  The
leader in the field has been NVIDIA with its CUDA
\cite{heinecke2012gpgpu} technology, but recently Intel joined the
market with its Many Integrated Cores (MIC) technology
\cite{heinecke2012gpgpu}. The efforts to unify development
across regular microprocessors and various accelerators led to OpenCL
\cite{heinecke2012gpgpu}, a set open standards for developing
applications that run across heterogeneous platforms.

The software development in scientific communities has steadily shifted
towards C/C++.  While there is still a lot of legacy code written in
Fortran (and hence continuing development), much of the new
development happens in C++ and Python \cite{python}. Examples are
Q-Chem \cite{qchem}, with most of its new development happening in
C++, and Psi4 \cite{psi4}, almost entirely implemented in C++ with
Python used as a scripting engine.  The C++ language and compilers
continue to evolve and improve at a faster pace than Fortran, mostly due
to influence of much larger commercial application development market.
In terms of raw speed, the C++ programs are as fast as Fortran
counterparts but C++ has the advantage of modern programming
techniques and many libraries and frameworks, e.g. Boost \cite{boost}.

So, what does the contemporary scientific computing platform look like
now?  It is almost always a distributed memory cluster of very fast
multicore computers, with between 2 and 64 GB of memory per
node.  Some clusters might have GPU accelerators to augment the
computational power. The number of cores in the cluster varies
greatly, from just a few to tens of thousands.  The interconnect can
be 1Gb Ethernet, an InfiniBand, of proprietary network, such as
SeaStar on Cray supercomputers.  The file system can be a local disk or
a parallel file system capable of storing terabytes of data.

Ultimately, it is the hardware (or rather the hardware limitations)
that dictates how the algorithm is to be designed.  Until we have
infinite memory and bandwidth, the algorithms will always have to be
designed with these limitations in mind.  Furthermore, the algorithms
have to be designed so as to account for a great variety of
system configurations.  Few general rules of thumb can be used as
general guidelines for designing scalable and efficient algorithms:
minimize communication, keep memory footprint low and introduce
adjustable parameters for memory use, use external libraries, e.g.
Linear Algebra Package \cite{lapack} (LAPACK), and make software easy
to modify, extend, and even rewrite, perhaps by using certain
programming language over another.  Furthermore, how will the
scientific computing landscape look in the future?  Who knows!  But we
better design the software so that changes dictated by the hardware
can be accommodated efficiently.

In the following chapters are attempts
to develop a modern, but simple and flexible, C++ foundation for
computational chemistry algorithms and several algorithm
implementations built upon that foundation with the above rules of
thumb in mind.

But before one can get into the intertwined details of science,
algorithms, and hardware some theoretical background is necessary to
explain to the reader in broad detail the basis sets, two-electron
integrals, and transformations which will form the bulk of the
subsequent pages.

\section{Hartree-Fock}
At the center of computational chemistry is the evaluation of the
time-independent Schrödinger equation eigenvalue problem,
$$ H\Psi = E\Psi $$
where $H$ is the Hamiltonian operator, $\Psi$ is the wavefunction
containing all the relevant information about the chemical system, $E$
is the energy of the system and eigenvalue of the Hamiltonian.  To be
a proper wavefunction, $\Psi$ must be square integrable and
normalized, $\braket{\Psi}{\Psi} = 1$, and antisymmetric to satisfy
the Pauli exclusion requirement for fermions.  The expectation value
$E$ then can be computed as:
$$\bra{\Psi} H \ket{\Psi} = E$$

In terms of individual contributions, the Schrodinger equation can be
written in terms of the kinetic and potential energies of the electrons and nuclei:
$$(T_{e} + T_{n} + V_{ee} + V_{en} + V_{nn})\Psi = E\Psi$$

$T_{e}$ and $T_{n}$ are the kinetic energy terms for electrons
and nuclei respectively
$$T_{e} = -\sum_e^{N_e} \frac{\nabla^2_e}{2} $$
$$T_{n} = -\sum_n^{N_n} \frac{\nabla^2_n}{2m_n} $$

$\nabla^2$ is the Laplacian operator,
$$\nabla^2 = \frac{\partial^2}{\partial x^2}  +
\frac{\partial^2}{\partial y^2} +
\frac{\partial^2}{\partial z^2}$$

$V_{ee}$ is the electron-electron repulsion term,
$$V_{ee} = \sum_{e < f}^{N_e} \frac{1}{r_{ef}}$$

$V_{en}$ is the electron-nucleus attraction term,
$$V_{ee} = -\sum_{e}^{N_e} \sum_n^{N_n} \frac{Z_n}{r_{en}}$$

$V_{nn}$ is the nucleus-nucleus repulsion term,
$$V_{nn} = \sum_{n < l}^{N_n} \frac{Z_n Z_l}{r_{nl}}$$



The closed form analytic solution for the Schrodinger equation
exists only for the simplest systems, such as those with one or two
particles.  To evaluate a quantum system of interest, a number of
approximations has to be made.  In the Born-Oppenheimer approximation
\cite{born1927quantentheorie} the much slower nuclei are treated as
stationary point charges and the Schrodinger equation then reduces to
the electronic Schrodinger equation:
$$H_e = T_e + V_{ee} + V_{en}$$
$$ H_e\Psi = E_e\Psi$$

The general problem of the type $\bra{\Psi} \frac{1}{r_{ij}} \ket{\Psi}$
has no analytic solution and further approximations must be made.  The
crudest solution is to assume that electrons do not interact with each
other. This leads to the independent particle model in which,
$$\Psi_{IPM} = \phi_1(\mathbf{r}_1)\phi_2(\mathbf{r}_2)...$$
is separable with respect to each electron coordinate vector.

The independent particle wavefunction does not satisfy the anti-
symmetry requirement, but properties of the determinant do (since
exchanging any two rows or columns changes the sign).
Taking the determinant of $\Psi_{IPM}$ leads to Slater
determinant $\Psi_{HF}$ which in turn leads to the Hartree-Fock method

$ \Psi_{HF}(\mathbf{x}_1, \mathbf{x}_2, \ldots, \mathbf{x}_N) =
\frac{1}{\sqrt{N!}} \left|
\begin{matrix}
    \psi_1(\mathbf{x}_1) & \psi_2(\mathbf{x}_1) & \cdots &
    \psi_N(\mathbf{x}_1) \\
    \psi_1(\mathbf{x}_2) & \psi_2(\mathbf{x}_2) & \cdots &
    \psi_N(\mathbf{x}_2) \\
    \vdots               & \vdots               &        & \vdots
    \\
    \psi_1(\mathbf{x}_N) & \psi_2(\mathbf{x}_N) & \cdots &
    \psi_N(\mathbf{x}_N)
\end{matrix} \right| $

After performing a series of algebraic manipulations, the closed-shell
Hartree-Fock energy can be written as

$$E_{HF} =
\bra{\Psi_{HF}} H_e \ket{\Psi_{HF}} =
2 \sum_i h_{ii} + \sum_{ij} (2 J_{ij} - K_{ij})$$

where $h_{ii}$ is the one-electron integral
$$h_{ij} =
( \psi_i | h_1 | \psi_j ) = \int { \psi_i
( -\frac{\nabla^2}{2} - \sum_n^{N_n} \frac{Z_n}{r_{n}} )
 \psi_j dr_1 } $$

and $J$ and $K$ terms, called Coulomb and exchange, respectively, are
two-electron integrals
$$J = (ij|ij)$$
$$K = (ij|ji)$$
$$(ij|kl) = \int \int \psi_i(r_1) \psi_j(r_1
) \frac{1}{r_{12}}
\psi_k(r_2) \psi_l(r_2) dr_1 dr_2 $$
 

From now on, the $HF$ label will be dropped and $\Psi$ will be
understood to refer to $\Psi_{HF}$ and $H$ to refer $H_e$.


The only constraint on the one particle orbitals is that they remain
orthonormal,
$$\braket{\psi_i} {\psi_j} = \delta_{ij}$$
Therefore the orbitals can be manipulated to affect energy.
According to the variational principle, the best orbitals are those that
minimize the energy,
$$\partial{E} = \frac{\partial \Psi}{\partial \psi} = 0$$
The method of Lagrange multipliers solves minimization problem with
constraints.  The resulting Lagrange equation
$$ \partial [
\bra{\Psi} H \ket{\Psi} -
\sum_{i,j} ( \lambda_{ij} \braket{\psi_i}{\psi_j} - \delta_{ij} )
] = 0$$
can be reduced to
$$F \psi_k = \lambda_{ij} \psi_k$$
where $F$ is the Fock operator
$$F = [h_1 + \sum_i (2 J_i - K_i)]$$
Taking the Lagrangian multipliers to be of the form
$$\lambda_{ij} = \delta_{ij} \epsilon_k$$
the Hartree-Fock minization problem becomes an eigenvalue problem:
$$F \psi_k = \epsilon_k \psi_k$$

Optimizing general orbitals in the above problem is not generally
feasible.  Instead Roothaan \cite{roothaan1951new} proposed to expand
orbitals in terms of a known basis and restrict optimization to
coefficients:
$$\psi_i = \sum_b^{N} c_{ib} \chi_b$$

The optimization of molecular orbitals $\psi_i$ in terms of a fixed
basis leads to the Hartree-Fock-Roothaan equations
$$FC = \epsilon SC$$
where $C$ is the coefficient matrix and $S = (\chi_i|\chi_j)$ is the
basis overlap matrix.  The above equation is almost a solvable
eigenvalue equation, except for the $S$ term.  Although a general
basis is not usually orthonormal, it can be orthonormalized in which
case the overlap matrix becomes the identity matrix, $S = \delta_{ij}$
and the Hartree-Fock-Roothaan equation takes the form of a regular
eigenvalue problem
$$FC = \epsilon C$$

Now an expression for the Fock operator can be derived in terms of the
coefficients and one- and two- electron integrals over basis functions
$$F = (\chi_i|h_1|\chi_j) +
      D [ (\chi_i \chi_j |h_2|\chi_k \chi_l) -
          (\chi_j \chi_k |h_2|\chi_j \chi_l) ]$$
where $D = 2 \sum_i c_{ib}*c_{ib}$ is known as density matrix.

Since the orbital coefficients appear on both sides of the equation,
the Hartree-Fock method needs to be repeated until the difference
between the old and the new coefficients reaches a certain threshold.
Because of that, the Hartree-Fock method is also called the self
consistent field (SCF) method.

The simple interpretation of the Hartree-Fock method is that an electron
is moving in the mean field of the other electrons.  The interaction of
individual electrons is not correlated, other than accounting for the
Pauli exclusion principle.  Accounting for electronic interaction will
be discussed below.

\section {Basis Set}
To understand the intricate details of the computational chemistry
algorithms, especially when discussing two-electron integrals, a few
words must be said about the basis set.

Modern basis sets are based on the atomic orbitals, which are
spatial orbitals reminiscent of the $s,p,...$ orbital shapes found in
physical chemistry books.  Because of that the basis sets are
often called atomic basis or atomic orbitals, as opposed to molecular
orbitals, which are simply atomic orbitals transformed via the
coefficient matrix $C$.

The correct shape for an (Cartesian) atomic orbital is the Slater-type
orbital (STO)
$$A x^l y^m z^n e^{-\alpha r}$$
where $A$ is the normalization coefficient and $l,m,n$ are related to
 the angular momentum quantum number $L$,
$$L = l + m + n$$

Using a Gaussian function, a similar type of orbital, called
Gaussian-type orbital (GTO), can be devised
$$A x^l y^m z^n e^{-\alpha r^2}$$

Unlike the Gaussian functions, the Slater functions cannot be
separated into $x,y,z$ components, making the evaluation of integrals
 over the Slater basis expensive.
On the other hand, Gaussian function can be written as
$$e^{-\alpha r^2} = e^{-\alpha x^2} e^{-\alpha y^2} e^{-\alpha z^2} $$
and due to this property, the computation of integrals over the
Gaussian functions is much simpler \cite{boys1950electronic}, with a
number of different closed-form solutions for one- and two- electron
integrals \cite{pople1978computation, rys_computation_1983,
  obara1986efficient, head1988method}.  Most electronic structure
programs use GTOs as basis sets.  An exception to this trend is
Amsterdam Density Functional (ADF) program suite \cite
{te2001chemistry} which uses STOs.

To reproduce the approximate shape of an STO, a linear
combination of several GTOs can be taken and fitted according to
some criteria, a process known as contraction and the resulting
orbital called contracted Gaussian-type orbital,
$$\chi_{cgto} = A x^l y^m z^n \sum_k^K C_k e^{-\alpha_k r^2}$$
where $K$ is the construction order and $C_k$ are the contraction
coefficients.  In this context, the individual Gaussians are called
primitives.


The individual contracted orbitals which share the same primitives are
grouped together into shells.  The primary reason for doing so is
computational efficiency.  With a correct algorithm, only the angular
term $x^l y^m z^n$ will be different between shell functions; the
terms involving expensive exponent computations will be the same.

The simplest contracted basis sets are of the STO-NG family
\cite{pople1970molecular}, where N is the number of contracted GTOs
fitted to an STO using a least-squares method.  The major difference
between GTOs and STOs is the function shape near the origin, where GTOs
are flat and STOs have a cusp.  This is especially important for the
core electrons near the nucleus.  More advanced basis sets typically
have more GTOs to represent contracted core orbitals (6-10 GTO) and
fewer GTOs to represent non-core orbitals (1-3 GTOs).  This segmented
approach strikes a delicate balance between accuracy and computational
time.

It should be obvious that a larger basis set will give better orbitals
and lower energy, based on the Variational Principle.  However, larger
basis set will also increase computational time, may lead to slower
convergence, and may result in numerical instabilities.  A majority of
time is spent evaluating two-electron integrals and building the Fock
matrix.  Although, atomic integrals do not change from iteration to
iteration, storing $N^4$ elements can be prohibitively expensive for any
large system, and thus the integrals can be re-computed on-the-fly.
Currently, Hartree-Fock computations with a few thousand basis
functions are routinely performed in a matter of hours.  In the near
future that number is likely to be the tens of thousands.

\section {Electron Correlation}
As a rule of thumb, the energy computed with the Hartree-Fock method
accounts for \~99 \% of the total electronic energy. However, the physical
properties associated with the last 1 \% of energy are what usually is
sought.  Hartree-Fock computations can give very good geometries, but
the energy differences can only be qualitative at best.

Recall from the above discussion that Hartree-Fock model does not
account for instantaneous electronic interaction, but instead treats
each electron as interacting with an electronic mean field.
The difference between the total energy and  the Hartree-Fock
energy is called the correlation energy
$$E_{corr} = E_{hf} - E$$

To recover the correlation energy, Hartree-Fock computations must be
followed by what are called correlation methods, which try to recover
the correlation energy from the Hartree-Fock wavefunction.  In the
context of electron correlation computations, Hartree-Fock is
typically the zeroth order (also called the reference) wavefunction.
Among the many correlation methods there are two that are central to
the next chapters: the MP2 and coupled cluster.

The formula for the MP2 energy is relatively simple,
expressed only in terms of molecular integrals $(ia|jb)$
and orbital energies $\epsilon$

$$E_{MP2} = \sum_{ij} \sum_{ab} 
\frac{ [2(ai|bj) - (bi|aj)] (ai|bj) }
{\epsilon_i + \epsilon_j - \epsilon_a - \epsilon_b}$$

As is customary in many-body methods, indices $i,j,...$ refer to
occupied molecular orbitals $O$, $a,b,...$ to virtual orbitals $V$,
and $p,q,r,s$ to atomic basis $N$. 

The time consuming part of the MP2 energy computation is not the
actual energy computation, which scales as $O^2N^2$, but the
transformation from atomic to molecular integrals (also called 4-index
transformation), which scale as $ON^4$.  Another bottleneck in
many-body methods is the storage of molecular integrals.  For a large
MP2 calculation the storage may well be on the order of terabytes.
The details of MP2 energy computation will be covered in detail in
the corresponding chapter.

The coupled cluster theory was first proposed in nuclear physics
\cite{coester1960short} and later adopted in quantum chemistry by
Cizek \cite{vcivzek1966correlation} as the exponential ansatz
$$\Psi = e^{T} \Psi_0 = e^{(T_1 + T_2 + ... T_n)} \Psi_0$$ 
where $T_1 ... T_n$ are the n-particle excitation operator and $\Psi_0$
is the reference wavefunction, typically $\Psi_{hf}$ in computational
chemistry.
The excitation operator applied to a reference wavefunction is written
in terms of excitation amplitudes $t$ from hole states $i,j,k,...$
(also referred to as occupied orbitals) to
particle states $a,b,c,...$ (also referred to as virtual orbitals),
$$T_n \Psi_0 = \sum_{ijk...} \sum_{abc...} t_{ijk...} ^ {abc...}
\Psi_{ijk...} ^ {abc...}$$

The CCSD algorithm is an iterative process that
scales as $N^2V^2O^2$ and the triples correction $(T)$ scales as
$N^2V^4O$.  To compute the CCSD(T) energy, every type of four-index
molecular integral is needed.  The coupled cluster algorithm will be
covered in detail in the last chapter.
Both, MP2 and CC can be easily and systematically derived using
Goldstone diagrams, a diagrammatic approach to nonrelativistic fermion
interaction based on Feynman diagrams.  A very thorough treatment of
the many-body theory can be found in the excellent book by Shavitt and
Bartlet \cite{shavitt2009many}.

%% \chapter{Two-Electron Integrals and Hartree-Fock}
%% \chapter{Moller-Plesset Perturbation Theory}
%% \chapter{Coupled Cluster Theory}

%% \chapter{Conclusions}


\bibliographystyle{unsrt}% (uses file ``plain.bst'')
\bibliography{references}
\documentclass{beamer}
\usepackage[latin1]{inputenc}
\usetheme{Warsaw}
\title[ Software Design In Computational Sciences]{Software Design In Chemistry}
\author{ Andrey Asadchev}
\institute{Iowa State University}
\begin{document}

\begin{frame}
\titlepage
\end{frame}


\begin{frame}{Outline}
\begin{itemize}
\item Programming Languages
\item Object Oriented Programming
\item C++ and Python
\item Symbolic Computation
\item Current Computer Architectures
\item Computational Chemistry
\item Integral Evaluation
\item Fock Matrix
\item GPU Implementation
\item Performance
\item Conclusions
\end{itemize}
\end{frame}

\begin{frame}{Programming languages}
\begin{itemize}
\item Communicate To Computer And To Humans
\item Binary Machine Language - instructions as sequence of zero and one
\item Assembly - mnemonic shortcuts to machine language
\item Fortran - early imperative programming language.\\
  Numerical and matrix computations.
\item LISP - functional programming language.\\
   Stands for List Processor.\\
   Functions (as in mathematics) are first-class citizens.\\
   To Iterate Is Human, To Recurse Is Divine.\\
   Lambda functions and predicates: $G = map(lambda\, x,y: x*y, F,reversed(G))$\\
   Artificial Intelligence\\
\item C - Portable Assembly Language.\\
  developed together with UNIX operating system.\\
  System Programming Language.\\
  Access to the raw memory.
\end{itemize}
\end{frame}



\begin{frame}{ Object-Oriented Programming}
\begin{itemize}
\item Abstract implementation problem 
\item Keep Data and Methods together
\item Hide Data, Puts Constraints on Data
\item Polymorphism 
\item Inheritance
\item Matrix\\
Represented As Some Object M\\
M.transpose()\\
const M.m\\
Orthogonal Matrix Is a Matrix\\
Orthogonal Matrix Has Transpose\\
Orthogonal Matrix  has inverse\\
\end{itemize}
\end{frame}


\begin{frame}{  C++}
\begin{itemize}
\item Superset of C
\item Object Oriented
\item Functional
\item Widely Used, Many Compilers
\item Efficient, games programming and digital signal processing
\item Overload operators, + , *,...
\item Program formulas almost like on the paper\\
$Vector\, r = v-w $\\
\item Template meta-programming\\
 Generic Programs\\
 Automatically Generated Programs\\
 Specialized Programs\\
 Boost\\
Domain Specific Language And Template Expressions\\
Readability: $Quartet \langle Shell \rangle$
\end{itemize}
\end{frame}

\begin{frame}{  Python}
\begin{itemize}
\item Interactive,compiled 
\item Imperative
\item Object Oriented
\item Functional
\item Easy interfacing with other languages
\item Pythonic\\
 L = sorted([(x,y) for x in X for y in Y if x is y])
\item Widely Used in Science and Mathematics\\
pyQuante
\item  Template-preprocessor engines (PHP)\\
  Cheetah\\
Preprocessor directives directly embedded in program\\
\end{itemize}
\end{frame}

\begin{frame}{ Symbolic computations}
\begin{itemize}
\item Computed Expressions Without Explicitly Knowing Number
\item Computer Algebra Systems
\item Mathematica.\\
  Polynomial Manipulation\\
Matrix Manipulation\\
Recursion\\
Allows to Define Your Own Algebra\\
\item  Sage.\\
    Python Computer Algebra System\\
    Interface to almost any other computer algebra system
\item Sympy - symbolic python library.\\
\item  Put Equations As They Are.\\
Let Computer Algebra System Simplify Them\\
Use Program Generator to produce code as you  wanted\\
Optimize equations which otherwise would be prohibitive to
\end{itemize}
\end{frame}

\begin{frame}{ Computer Architecture}
\begin{itemize}
\item Pipeline
\item Single Instruction Multiple Data
\item Memory Locality
\item Many cores
\item Complex Scheduling
\item Very Hard to Beat a Compiler for Low-Level Optimization\\
however programs have to be written in such way as to be amenable to optimization\\
Conditional statements, unpredictable memory access prohibit optimization.
\item Processors Are Powerful and Cheap\\
     Optimized for Games and Signal Processing, 4x4 matrix\\
     Underutilized often, 15 percent efficiency is common
\item  Efficient Implementations demand unrolling kernels
\end{itemize}
\end{frame}

\begin{frame}{ Computer Architecture}
\begin{itemize}
\item 
\item   Graphical Units\\
 A Lot Of Performance And Even More Hype\\
 Memory Bound Often\\
 Single Instruction Multiple Thread\\
 Imagine mapping each loop iteration to threads\\
 for i  = threadID,N,numThreads:  X(i) = aY(i)
\end{itemize}
\end{frame}

\begin{frame}{  Computational Chemistry}
\begin{itemize}
\item Electron Integrals $(ab|cd)$\\
  Domain Specific\\
  Define rest of computations
\item Lots of Computations, Lots of Memory
\item Data Screening, Symmetry
\item  Linear Algebra, BLAS
\item Legacy Code
\item MPQC, libint
\end{itemize}
\end{frame}

\begin{frame}{   Integral  evaluation}
\begin{itemize}
\item $(ab|cd) = \sum C_i \sum C_j \sum C_k \sum C_l [ij|kl] $
\item $(ab|cd) = (ba|cd) = (cd|ab)...$
\item many different combinations and permutations
\item SP functions
\item high angular momentum versus high confection order\\
$(fsp | fsp)$
\item gaussian basis, $e ^{-r^2}$ separable in Cartesian coordinates
\item  Most Integral schemes use recursive nature of gaussian integration/differentiation
\item Numerical error - loss of significant figures, etc.\\
Recursion implies using a slightly incorrect value to compute next\\
Especially severe when two values are close in magnitude but different in sign (difference)
\end{itemize}
\end{frame}

\begin{frame}{   Integral  evaluation}
\begin{itemize}
\item 
\item OS integral scheme, auxiliary integrals. \\
Very general - applicable to almost any operator\\
R12 linear integrals\\
Memory Hungry
\item Rys quadrature\\
  Gaussian quadrature using Rys orthogonal polynomials\\
  $I = \sum_a Ix(a)Iy(a)Iz(a)$\\
  Compute Roots\\
  Compute two-dimensional x, y, z intermediate integrals\\
  Assemble final integral
\item Horizontal recurrence\\
$(pp | = q*(ps |  + (ds |$\\
Simplifies Contracted Integrals\\
High Numerical Error
\end{itemize}
\end{frame}

\begin{frame}{  Integral Implementation}
\begin{itemize}
\item C++ object-oriented library
\item Fortran And Gamess bindings
\item  Simple Interface:\\
Create quadrature object for given shell quartet\\
Call Object operator for a given center quartet
call object operator for a list of Center quartets

\item Two Internal Implementations:\\
  Fully Unrolled And Simplified kernels for smaller  integral, 
  which is likely to be contracted\\
  Partially Unrolled (bra only) general quadrature\\
\item Internals make heavy use of  C++ templates and automatically generated code
\item Human Serviceable Code is small - around 2000 lines of code
\item Final Library Size his small - around 5 megabytes fully optimized
\item the code is kept small due to objects and genetic templates
\end{itemize}
\end{frame}

\begin{frame}{  Fock Matrix}
\begin{itemize}
\item $F = (ab |cd)D $
\item Part of the Library
\item Requires Matrix to be in block form through an adapter
\item Simple Interface:\\
Create fock object for given shell quartet\\
Call Object operator for a given center quartet\\
Call Object operator for a list of Center quartets
\item Two Internal Implementations - parallel to those of integral program
\item Basis Set organization and higher-level logic is handled by another library
\end{itemize}
\end{frame}

\begin{frame}{  GPU implementation}
\begin{itemize}
\item Implementations are driven by integral size and contraction order
\item Large integrals are parallel over individual elements\\
every thread gets a unique internal element to compute
\item highly contracted small integrals are parallel over contractions\\
every thread gets the unique contraction to compute\\
contractions are  reduced in the end
\item Basis  must be ordered to guarantee execution of the kernel on multiple centers
\item Thanks to objects and genetic templates, the entire kernel code is about 300 lines
\end{itemize}
\end{frame}

\begin{frame}{   Performance}
\begin{itemize}
\item GPU implementation not yet complete
\item CPU implementation:\\
  benchmark several cases from hundred to a thousand basis functions\\
  largest test is loperamide 6-31G(pdf)
  numerical agreement to 8 decimal places\\
  30-40\% improvement in speed\\
  Fine grain contraction screening was not enabled\\
  Intel compiler generates vector instructions\\
  still working on full optimization
\item GPU implementation not yet complete\\
  basic integrals, up to the 1500 quartet size are working\\
  50\% improvement over CPU\\
  numbers essentially agree
\end{itemize}
\end{frame}


\begin{frame}{Aknowledge}
\begin{itemize}
\item Funding From Dr. Gordon
\item  Help From Jacob and Allada
\end{itemize}
\end{frame}

\end{document}


\end{document}



% LocalWords:  
