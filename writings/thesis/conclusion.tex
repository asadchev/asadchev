\documentclass[12pt]{article} \usepackage[margin=1in]{geometry}

\usepackage{graphicx}
\usepackage{color}

%% \usepackage{amsmath}
%% \usepackage{listings}

%% \lstloadlanguages{C++} \lstset{ language=C++, breaklines=true,
%%   keywordstyle=\color{blue}, commentstyle=\color{red} }


\newenvironment{listing}%
               {\begin{table}
                   \begin{tabular}{ p{6in} }
                     \hline}%
               {\end{tabular}%
               \end{table}}


\begin{document}

\title{A New Algorithm for Second Order Perturbation Theory}
\author{Andrey Asadchev \and Mark S. Gordon}
\date{}

\maketitle
% \abstract{}

\section{Conclusions}
As the computing technology changes and matures, the scientific
computing must follow.  The hardware and software which was cutting
edge in the 70's and 80's still dictates how many of the computational
chemistry packages are still implemented today.  However, the
computing technology evolved very quickly since the Fortran 77.
Object oriented programming (OOP), generic programming, standard
libraries, and system standards have become the essential pieces of
most modern  commercial and open-source software, small and large
alike.  To keep up with the improvements in computer science, the
computational chemistry algorithms must be either modernized or
rewritten.  Often, due to software architecture decisions made decades
ago, rewriting is the only viable plan for the future.
Not all of the software needs to be  modernized at once:  the key
pieces such as integral and Hartree-Fock methods can be rewritten
alone and integrated into the existing software, one at a time.
Software modernization also presents a opportunity to improve the
existing algorithms, separate them into modular libraries to encourage
reuse among the scientists, and to plan ahead, given the trends in
computing over the last few decades.


The first algorithm presented was the Hartree-Fock, the reference
method in the most election correlation theories.  The Hartree-Fock
method requires evaluation of the two-electron integrals, which
constitutes the most consuming part.  Unlike other pieces in the
computational chemistry, two-electron integral methods are specific to
the domain and do not receive much attention from outside the field.
The integrals were implemented using Rys Quadrature, one out of
several integral methods.  While algorithmically more complex than
other methods, Rys Quadrature is a general numerically stable method
with low memory footprint, which makes it suitable for implementation
on graphical processing units (GPU).  Once the integral engine was
implemented, the multithreaded Hartree-Fock method naturally followed.
The integral and Hartree-Fock GPU implementation was able to reuse
many key pieces of the CPU algorithm, designed to be fast, extensible,
and flexible through the use of code generator and C++ templates.


One of the most common electron correlation methods is second order
many-body perturbation theory, also known as Moller-Plesset second
order perturbation theory (MP2).  Unlike higher-order treatments,
MP2 is relatively inexpe black-box method  which makes it very
popular.  Hence, the Hartree-Fock implementation was followed by MP2
method.  Like the Hartree-Fock method, the MP2  implementation relies
heavily on the fast integrals.  But unlike the Hartree-Fock, most of
computational work is handled by the de facto standard basic linear algebra
subroutines, BLAS.   The MP2 algorithm implemented is a semi-direct
method, meaning that the partially transformed integrals need to be stored
in the secondary storage, such as disk or distributed memory.   Unlike
the other MP2 algorithms, which are either disk or distributed
memory, the implemented algorithm uses OOP features of C++ to provide
transparent integral storage on either disk or in  distributed memory.

The natural follow-up to MP2 is the coupled cluster (CC) theory.
The couple cluster truncated at singles and doubles excitations, CCSD,
with the perturbative triples correction (T) leads to CCSD(T) method,
often called the gold standard of computational chemistry due its
accuracy.  The CCSD(T)  is very expensive method, both in terms of
computer time and memory.  However, with the lessons learned designing
the MP2 algorithm, a fast CCSD(T) algorithm was developed such that it
could run equally a single workstations and supercomputers.
The key to implementation was optimizing the algorithm in terms of
memory first, I/O overhead second, and concentrating on the
computational efficiency last.  By using several properties of atomic
to molecular basis transformations, several expensive computation and
storage requirements were eliminated from the CCSD algorithm.  And by
using well-known loop optimization technique called blocking, the (T)
algorithm was implemented with very little memory and very little I/O
overhead.

The three algorithms summarized above were prompted by
the need to accommodate the wide array of computational hardware.  In
the process, the algorithms were improved, often drastically.
Implemented in C++, the algorithms and the supporting framework were
built as a stand-alone library, with Fortran bindings.  Connected to
GAMESS, the library was successively integrated with the existing
legacy code.  While not explicitly discussed, the supporting
framework, such as basis set and wavefunction objects, is absolutely
necessary to develop robust flexible modern code.


% \bibliographystyle{unsrt}% (uses file ``plain.bst'')
% \bibliography{references}
% \documentclass{beamer}
\usepackage[latin1]{inputenc}
\usetheme{Warsaw}
\title[ Software Design In Computational Sciences]{Software Design In Chemistry}
\author{ Andrey Asadchev}
\institute{Iowa State University}
\begin{document}

\begin{frame}
\titlepage
\end{frame}


\begin{frame}{Outline}
\begin{itemize}
\item Programming Languages
\item Object Oriented Programming
\item C++ and Python
\item Symbolic Computation
\item Current Computer Architectures
\item Computational Chemistry
\item Integral Evaluation
\item Fock Matrix
\item GPU Implementation
\item Performance
\item Conclusions
\end{itemize}
\end{frame}

\begin{frame}{Programming languages}
\begin{itemize}
\item Communicate To Computer And To Humans
\item Binary Machine Language - instructions as sequence of zero and one
\item Assembly - mnemonic shortcuts to machine language
\item Fortran - early imperative programming language.\\
  Numerical and matrix computations.
\item LISP - functional programming language.\\
   Stands for List Processor.\\
   Functions (as in mathematics) are first-class citizens.\\
   To Iterate Is Human, To Recurse Is Divine.\\
   Lambda functions and predicates: $G = map(lambda\, x,y: x*y, F,reversed(G))$\\
   Artificial Intelligence\\
\item C - Portable Assembly Language.\\
  developed together with UNIX operating system.\\
  System Programming Language.\\
  Access to the raw memory.
\end{itemize}
\end{frame}



\begin{frame}{ Object-Oriented Programming}
\begin{itemize}
\item Abstract implementation problem 
\item Keep Data and Methods together
\item Hide Data, Puts Constraints on Data
\item Polymorphism 
\item Inheritance
\item Matrix\\
Represented As Some Object M\\
M.transpose()\\
const M.m\\
Orthogonal Matrix Is a Matrix\\
Orthogonal Matrix Has Transpose\\
Orthogonal Matrix  has inverse\\
\end{itemize}
\end{frame}


\begin{frame}{  C++}
\begin{itemize}
\item Superset of C
\item Object Oriented
\item Functional
\item Widely Used, Many Compilers
\item Efficient, games programming and digital signal processing
\item Overload operators, + , *,...
\item Program formulas almost like on the paper\\
$Vector\, r = v-w $\\
\item Template meta-programming\\
 Generic Programs\\
 Automatically Generated Programs\\
 Specialized Programs\\
 Boost\\
Domain Specific Language And Template Expressions\\
Readability: $Quartet \langle Shell \rangle$
\end{itemize}
\end{frame}

\begin{frame}{  Python}
\begin{itemize}
\item Interactive,compiled 
\item Imperative
\item Object Oriented
\item Functional
\item Easy interfacing with other languages
\item Pythonic\\
 L = sorted([(x,y) for x in X for y in Y if x is y])
\item Widely Used in Science and Mathematics\\
pyQuante
\item  Template-preprocessor engines (PHP)\\
  Cheetah\\
Preprocessor directives directly embedded in program\\
\end{itemize}
\end{frame}

\begin{frame}{ Symbolic computations}
\begin{itemize}
\item Computed Expressions Without Explicitly Knowing Number
\item Computer Algebra Systems
\item Mathematica.\\
  Polynomial Manipulation\\
Matrix Manipulation\\
Recursion\\
Allows to Define Your Own Algebra\\
\item  Sage.\\
    Python Computer Algebra System\\
    Interface to almost any other computer algebra system
\item Sympy - symbolic python library.\\
\item  Put Equations As They Are.\\
Let Computer Algebra System Simplify Them\\
Use Program Generator to produce code as you  wanted\\
Optimize equations which otherwise would be prohibitive to
\end{itemize}
\end{frame}

\begin{frame}{ Computer Architecture}
\begin{itemize}
\item Pipeline
\item Single Instruction Multiple Data
\item Memory Locality
\item Many cores
\item Complex Scheduling
\item Very Hard to Beat a Compiler for Low-Level Optimization\\
however programs have to be written in such way as to be amenable to optimization\\
Conditional statements, unpredictable memory access prohibit optimization.
\item Processors Are Powerful and Cheap\\
     Optimized for Games and Signal Processing, 4x4 matrix\\
     Underutilized often, 15 percent efficiency is common
\item  Efficient Implementations demand unrolling kernels
\end{itemize}
\end{frame}

\begin{frame}{ Computer Architecture}
\begin{itemize}
\item 
\item   Graphical Units\\
 A Lot Of Performance And Even More Hype\\
 Memory Bound Often\\
 Single Instruction Multiple Thread\\
 Imagine mapping each loop iteration to threads\\
 for i  = threadID,N,numThreads:  X(i) = aY(i)
\end{itemize}
\end{frame}

\begin{frame}{  Computational Chemistry}
\begin{itemize}
\item Electron Integrals $(ab|cd)$\\
  Domain Specific\\
  Define rest of computations
\item Lots of Computations, Lots of Memory
\item Data Screening, Symmetry
\item  Linear Algebra, BLAS
\item Legacy Code
\item MPQC, libint
\end{itemize}
\end{frame}

\begin{frame}{   Integral  evaluation}
\begin{itemize}
\item $(ab|cd) = \sum C_i \sum C_j \sum C_k \sum C_l [ij|kl] $
\item $(ab|cd) = (ba|cd) = (cd|ab)...$
\item many different combinations and permutations
\item SP functions
\item high angular momentum versus high confection order\\
$(fsp | fsp)$
\item gaussian basis, $e ^{-r^2}$ separable in Cartesian coordinates
\item  Most Integral schemes use recursive nature of gaussian integration/differentiation
\item Numerical error - loss of significant figures, etc.\\
Recursion implies using a slightly incorrect value to compute next\\
Especially severe when two values are close in magnitude but different in sign (difference)
\end{itemize}
\end{frame}

\begin{frame}{   Integral  evaluation}
\begin{itemize}
\item 
\item OS integral scheme, auxiliary integrals. \\
Very general - applicable to almost any operator\\
R12 linear integrals\\
Memory Hungry
\item Rys quadrature\\
  Gaussian quadrature using Rys orthogonal polynomials\\
  $I = \sum_a Ix(a)Iy(a)Iz(a)$\\
  Compute Roots\\
  Compute two-dimensional x, y, z intermediate integrals\\
  Assemble final integral
\item Horizontal recurrence\\
$(pp | = q*(ps |  + (ds |$\\
Simplifies Contracted Integrals\\
High Numerical Error
\end{itemize}
\end{frame}

\begin{frame}{  Integral Implementation}
\begin{itemize}
\item C++ object-oriented library
\item Fortran And Gamess bindings
\item  Simple Interface:\\
Create quadrature object for given shell quartet\\
Call Object operator for a given center quartet
call object operator for a list of Center quartets

\item Two Internal Implementations:\\
  Fully Unrolled And Simplified kernels for smaller  integral, 
  which is likely to be contracted\\
  Partially Unrolled (bra only) general quadrature\\
\item Internals make heavy use of  C++ templates and automatically generated code
\item Human Serviceable Code is small - around 2000 lines of code
\item Final Library Size his small - around 5 megabytes fully optimized
\item the code is kept small due to objects and genetic templates
\end{itemize}
\end{frame}

\begin{frame}{  Fock Matrix}
\begin{itemize}
\item $F = (ab |cd)D $
\item Part of the Library
\item Requires Matrix to be in block form through an adapter
\item Simple Interface:\\
Create fock object for given shell quartet\\
Call Object operator for a given center quartet\\
Call Object operator for a list of Center quartets
\item Two Internal Implementations - parallel to those of integral program
\item Basis Set organization and higher-level logic is handled by another library
\end{itemize}
\end{frame}

\begin{frame}{  GPU implementation}
\begin{itemize}
\item Implementations are driven by integral size and contraction order
\item Large integrals are parallel over individual elements\\
every thread gets a unique internal element to compute
\item highly contracted small integrals are parallel over contractions\\
every thread gets the unique contraction to compute\\
contractions are  reduced in the end
\item Basis  must be ordered to guarantee execution of the kernel on multiple centers
\item Thanks to objects and genetic templates, the entire kernel code is about 300 lines
\end{itemize}
\end{frame}

\begin{frame}{   Performance}
\begin{itemize}
\item GPU implementation not yet complete
\item CPU implementation:\\
  benchmark several cases from hundred to a thousand basis functions\\
  largest test is loperamide 6-31G(pdf)
  numerical agreement to 8 decimal places\\
  30-40\% improvement in speed\\
  Fine grain contraction screening was not enabled\\
  Intel compiler generates vector instructions\\
  still working on full optimization
\item GPU implementation not yet complete\\
  basic integrals, up to the 1500 quartet size are working\\
  50\% improvement over CPU\\
  numbers essentially agree
\end{itemize}
\end{frame}


\begin{frame}{Aknowledge}
\begin{itemize}
\item Funding From Dr. Gordon
\item  Help From Jacob and Allada
\end{itemize}
\end{frame}

\end{document}


\end{document}



% LocalWords:  
