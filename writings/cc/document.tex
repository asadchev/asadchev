\documentclass[12pt]{article} \usepackage[margin=1in]{geometry}

\usepackage{graphicx}
\usepackage{color}
\usepackage{amsmath}

%% \usepackage{amsmath}
%% \usepackage{listings}

%% \lstloadlanguages{C++} \lstset{ language=C++, breaklines=true,
%%   keywordstyle=\color{blue}, commentstyle=\color{red} }

\newcommand{\bra}[1]{\langle #1|}
\newcommand{\ket}[1]{|#1\rangle}
\newcommand{\braket}[2]{\langle #1|#2\rangle}

\newenvironment{listing}%
               {\begin{table}
                   \begin{tabular}{ p{6in} }
                     \hline}%
               {\end{tabular}%
               \end{table}}


\begin{document}

\title{Another CCSD(T) Implementation}
\author{Andrey Asadchev \and Mark S. Gordon}
\date{}

\maketitle


\abstract{ A new coupled cluster singles and doubles with triples
correction, CCSD(T), algorithm is presented. The new
algorithm is implemented in C++, has low memory footprint, fast
execution time, low I/O overhead, and flexible storage backend with
the ability to use either distributed memory or filesystem for
 storage.  The algorithm is demonstrated to work well on
single workstation, small cluster, and a high-end Cray computer.\ With
the new implementation, a CCSD(T) calculation with several hundred
basis functions and a few dozen occupied orbitals can run in under a
day on a single workstation.}


\section{Introduction}
As a rule of thumb, the electronic energy obtained through
Hartree-Fock method accounts for the 99 \% of the energy.  However,
the chemical properties of interest are dependent on the remaining 1
\%.  The difference between the reference Hartree-Fock energy and the
true energy is defined as correlation energy,
$$E_{corr} = E_{hf} - E$$

Of the many electron correlation methods \cite{pople1977variational,
pople1987quadratic,pople1992kohn}, the coupled cluster method with
perturbative triples correction \cite{raghavachari1989fifth}, CCSD(T),
is the most successful, often referred to as the gold standard of the
computational chemistry.  The coupled cluster method was first
developed by nuclear physicists \cite{coester1960short}, adopted to
quantum chemistry by Cizek \cite{vcivzek1966correlation,paldus1972correlation,mukherjee1977applications,lindgren1978coupled,scuseria1988efficient,piecuch1999coupled}, and
popularized by Bartlett \cite{purvis1982full}.

The coupled cluster method usually introduced in the exponential
{\it ansatz} form \cite{szabo1996modern, janowski2007parallel},
$$\Psi = e^{T} \Psi_0 = e^{(T_1 + T_2 + ... T_n)} \Psi_0$$ 
where $T_1 ... T_n$ are the n-particle cluster operator and $\Psi_0$
is the reference wavefunction, typically the Hartree-Fock reference
$\Psi_{hf}$.

The excitation operator applied to a reference wavefunction is written
in terms of cluster excitation amplitudes $t$ from hole states $i,j,k,...$
(occupied orbitals in chemistry parlance) to particle states 
(or virtual orbitals), $a,b,c,...$
$$T_n \Psi_0 = \sum_{ijk...} \sum_{abc...} t_{ijk...} ^ {abc...}
\Psi_{ijk...} ^ {abc...}$$

Truncating the expansion at doubles, leads to coupled cluster singles
and double method, CCSD,
$$T \approx T _1 + T_2$$
The $t^a_i$ and $t^{ab}_{ij}$ are found by solving a system of nonlinear
equations,
$$\bra{\Phi^a_i} (H_N e^{T_1 + T_2}) \ket{\Phi} = 0$$
$$\bra{\Phi^{ab}_{ij}} (H_N e^{T_1 + T_2}) \ket{\Phi} = 0$$
where $H_N = H - \bra{\Phi} H \ket{\Phi}$ is the normal order
Hamiltonian \cite{yarkony1995modern}

The final algebraic CC equations, derived using diagramatic
approach, result in a number of integral terms $V$ contracted with
$T$ amplitudes, e.g. $VT_1^2$ signifies integral terms contracted with
$t_i^a t_j^b$.  The complete derivation can be found in the number of
sources \cite{shavitt2009many}.
For the purposes of this work, the spin-free equations by
Piecuch \cite{Piecuch2002efficient} are used.

As is customary, the algebraic CC equations are presented in Eistein
summation, where repeated co- and contra-variant index implies
summation.
In terms of one-electron integrals $f^p_q = \bra{p} f \ket{q}$ and
two-electron molecular integrals $v$, the CCSD non-linear equations are,

% \documentclass[12pt]{article}

% \usepackage[margin=1in]{geometry}
% \usepackage{amsmath}

% %% \usepackage{amsmath}
% \usepackage{color}
% \usepackage{listings}
% \usepackage{graphicx}

% \begin{document}

% \lstloadlanguages{C++}
% \lstset{
%   language=C++, 
%   breaklines=true,
%   keywordstyle=\color{blue},
%   commentstyle=\color{red}
% }

% %\maketitle 

\begin{align}
  \begin{split}
    D^a_i t^a_i &=
    f^a_i + I^a_e t^e_i - I{'}^m_i t^a_m + I^m_e (2 t^{ea}_{mi} - t^{ea}_{im}) + t^e_m (2 v^{ma}_{ei} - v^{am}_{ei}) \\
    & v^{mn}_{ei} (2 t^{ea}_{mn} - t^{ae}_{mn}) + v^{ma}_{ef}(2 t^{ef}_{mi} - t^{ef}_{im})
  \end{split}
  \label{t1}
\end{align}

\begin{align}
  \begin{split}
    D^{ab}_{ij} t^{ab}_{ij} &=
    v_{ij} + P (ia/j b) [ t^{ae}_{ij} I^b_e - t^{ab}_{im} I^m_j + \frac{1}{2} v^{ab}_{ef} c^{ef}_{ij} +  \frac{1}{2} c^{ab}_{mn} I^{mn}_{ij} \\
      & - t_{mj}^{ae} I_{ie}^{mb} - I_{ie}^{ma} t_{mj}^{eb} + (2t^{ea}_{mi} - t_{im}^{ea}) I_{ej}^{mb} + t_i^e I{'}_{ej}^{ab} - t_m^a I{'}_{ij}^{mb}] 
  \end{split}
  \label{t2}
\end{align}

\rule{\linewidth}{0.1mm}

\begin{align}
%
I^i_a &= f^i_a + 2 v^{im}_{ae} t^{e}_{m} - v^{im}_{ea} t^{e}_{m} \\
%
I^a_b &= (1 - \lambda^b_a) f^a_b +
         (2v^{am}_{be} t^e_m - v^{ma}_{be} t^e_m) -
         (2v^{mn}_{eb} c^{ea}_{mn} - v^{mn}_{be} c^{ea}_{mn}) -
         t^a_m f^m_b \\
%
I^i_j &= I'^i_j + I^i_e t^{e}_{j} \\
%
I'^i_j &= (1 - \lambda^i_j)f^i_j +
          (2v^{im}_{je} t^e_m - v^{im}_{ej} t^e_m) +
          (2v^{mi}_{ef} t^{ef}_{mj} - v^{im}_{ef} t^{ef}_{mj}) \\
%
I_{kl}^{ij} &= v^{ij}_{kl} + v^{ij}_{ef} c^{ef}_{kl} + P(ik/jl)t^e_k v^{ij}_{el} \\
%
I_{jb}^{ia} &= v^{ia}_{jb} - \frac{1}{2} v^{im}_{eb} c^{ea}_{jm} - v^{im}_{jb}t^a_m + v^{ia}_{eb}t^e_j \\
%
I{'}^{ab}_{ci} &= v^{ab}_{ci} - v^{am}_{ci} t^b_m - t^a_m v^{mb}_{ci} \\
I{'}^{ia}_{jk} &= v^{ia}_{jk} + v^{ia}_{ef} t^{ef}_{jk} + t^e_j t^f_k v^{ia}_{ef}
\end{align}


%\end{document}



% LocalWords:  Fock un def Multithreaded mutex ERI Submatrix bst


where $D$ are the many-body denominators
$$D^{p...}_{q...} = f^q_q - f^p_p + ...$$
and $P$ is the permutation operator,
$$P(ia/jb)u^{ij}_{ab} = u^{ij}_{ab} + u^{ji}_{ba}$$
Note that the permutation operator, akin to $A + A^{'}$, has the effect
of symmetrizing the operand such that
$$P(ia/jb)u^{ij}_{ab} = P(ia/jb)u^{ji}_{ba}$$

The molecular integrals are obtained from atomic orbital (AO) basis
via 4-index transformation,
$$v^{ab}_{cd} = C^a_p C^b_r C_c^q C_d^s \bra{pq} \frac{1}{r}
\ket{rs}$$
The transformed integrals have the following general symmetries,
$$v_{ar}^{bs} = v_{br}^{as}$$
$$v_{ab}^{qs} = v_{ba}^{sq}$$

The (T) correction is given as:
\begin{align}
  E^{[T]} &= \bar{t}^{ijk}_{abc} t^{abc}_{ijk} D^{abc}_{ijk} \\
  E^{(T)} &= E^{[T]} + t^{abc}_{ijk} D^{abc}_{ijk} \bar{z}^{ijk}_{abc}
  \label{(t)}
\end{align}
where
$$\bar{x}^{ijk}_{abc} = \frac{4}{3} x^{ijk}_{abc} - 2x^{ijk}_{acb} +
\frac{2}{3} x^{ijk}_{bca}$$
$$z^{ijk}_{abc} = (t^i_a v^{jk}_{bc} + t^j_b v^{ik}_{ac} + t^k_c v^{ij}_ab)/D^{abc}_{ijk}$$
and the $T_3$ amplitudes are,
\begin{align}
  \begin{split}
    D^{abc}_{ijk} t^{abc}_{ijk} &=
    P(ia/jb/kc) [ t^{ae}_{ij} v^{bc}_{ek} - t^{ab}_{im} v^{mc}_{jk}] 
  \end{split}
  \label{t3}
\end{align}
where,
$$P(ia/jb/kc)u^{ijk}_{abc} =
u^{ijk}_{abc} + u^{ikj}_{acb} +
u^{jik}_{bac} + u^{jki}_{bca} +
u^{kji}_{cba} + u^{kij}_{cab}$$

\subsection{Computational details}

CCSD equations are non-linear and must be solved to self-consistenct
via iterative procedure, usually with a help of acceleration method
\cite{scuseria1986accelerating}.  The CCSD method is dominated by its
most expensive term, $v^{ab}_{ef} c^{ef}_{ij}$, which scales as
$v^4o^2$.  The algorithm is expensive in terms of memory and storage
as well, with amplitude storage on the order of $v^2o^2$ and integrals
storage on the order of $v^4, v^3o, ...$ and so on.  The amount of the
in-core memory depends on the algorithm; most of algorithms require
$v^2o^2$ storage per node.  This amount of memory is non scalable,
e.g. 100 occupied and 1000 virtual orbitals require 80GB of memory per
node which is very uncommon.

The (T) is non-iterative correction requires $v^3o$ storage and scales
as $v^4o^3$.  A naive (T) algorithm is trivial to implement but an
algorithm with a small memory and scalable I/O requires thought.

There is a CCSD(T) method in nearly every quantum chemistry package.
The state-of-the-art is ACES \cite{lotrich2008parallel}; its
implementation can handle very large systems on a large cluster.
NWChem \cite{nwchem} implementation is slower but is similarly
scalable.  The MOLPRO \cite{molpro} algorithm has $o^2v^2$ memory
requirement, which limits its utility, but it is perhaps the fastest
there is for smaller calculations.  The GAMESS \cite{gamess}
implementation runs in parallel but is similarly limited by $o^2 v^2$
memory requirement.

Only two implementations, the NWChem and ACES, are able to
handle computations of the order of thousand basis functions, but both
require access to fairly large machines to do so.  See for example
ACES Parallel Coupled Cluster benchmarks \cite{pcc}.

\section{Design of a Scalable and Efficient Algorithm}
Previously, we reported an MP2 energy algorithm, which has small
memory footprint, good performance, flexible storage implementation,
and is able to run on workstations and clusters equally well.  In the
same spirit, a coupled cluster algorithm can be designed, such that it
is efficient, has small memory footprint, is able to utilize filesystem
and memory for storage, and as a result can run on machines with very
different capabilities.

As far as the coupled cluster algorithms (and other many-body methods)
are concerned, it is the memory that is most likely to limit the
application of the algorithm.  Memory is a limited resource, unlike
the time.  Furthermore, calculations can be sped up by throwing more
computational hardware at a problem, whereas the amount of physical
memory per node can not be increased by adding another node.

Some very large arrays can be (and need to be) distributed across the nodes
(distributed memory) or stored on the filesystem.  The disks are cheap
and offer terabytes of storage, but filesystem I/O can be very slow if
not done right.  Nevertheless, a considerable amount of memory must be
present to carry out local calculations.

What are the memory limitations of the current hardware?  On a
{\it typical} workstation or a cluster node, there is anywhere between
1GB and 8GB per core, with 2GB of RAM (or less) {\it much} more common
than, say, 4GB per {\it core}.  Per entire {\it node}, the amount of
memory can vary from 2GB up to 64GB and more, with 8-16GB fairly
common.  That number will increase in the future, but at much slower
rate than the increase in computational power.

To draw a connection between memory and the dimensions present in the CC
calculations, several arrays, corresponding to 100 occupied
orbitals and 1000 and 2000 basis functions, are listed in Table
\ref{limits}.  It should be kept in mind that the sizes listed are not
for the entire calculation, but for one of the several arrays needed.
Some of the arrays can be shared, but some must be allocated per
thread/core.

it should be clear that storing $o^2n^2$ quantity per node (let alone
per core) is too expensive: a node with 80GB of RAM is rare and one
with  320GB even more so.  The same goes for the quartic arrays other
than $o^4$ and arrays involving $n^2$ factor: storing for example
several 3GB arrays would preclude most systems from being able to
handle more than a thousand basis functions. In the end we are left
with restricting memory requirements to $o^2n$ (or smaller) arrays,
which size only increases linearly with basis set.
Trying to limit memory further than $o^2n$, to say $on$, will come at
the very high cost of increased I/O.

\begin{table}
  \label{limits}
  \caption {Array Sizes for o=100}
  \begin{center}
    \begin{tabular}{| l | c | c |}
      \hline
      Array     & Size (GB), n=1000 & Size (GB), n=2000 \\
      \hline
      $o^4$     & 0.8               & 0.8               \\ 
      $o^2n$    & 0.8               & 1.6               \\
      \hline
      $on^2$    & 0.8               & 3.2              \\
      $o^3n$    & 8.0               & 16.0              \\
      $n^3$     & 8.0               & 64.0              \\
      $o^2n^2$  & 80.0              & 320.0             \\
      \hline
      $on^3$    & 800.0             & 6400.0             \\
      $n^4$     & 8000.0            & 128000.0             \\
      \hline
    \end{tabular}
  \end{center}
\end{table}

Some arrays, notably $n^4$, are too great to store even in the
secondary storage.  The terms involving that array need to be
evaluated directly, i.e.on the fly,  at the modest cost of recomputing
atomic integrals, cf. \cite{olson2007novel}.
However, to push the ability of the algorithm beyond a
thousand basis function, $on^3$ storage also needs to be eliminated in
the CCSD algorithm.
To ensure that I/O overhead is low even on filesystems, transfers to
and from secondary storage must be contigous and in large chunks.
There are three basic remote operations: {\tt put, get, accumulate}.
The last one cannot be implemented efficiently via
the filesystem I/O and the algorithm must not rely on it.

Finally, to achieve computational efficiency, all the expensive tensor
contractions need to be carried out using {\tt dgemm} and tensor
permutations must not exceed two adjecent indices, e.g.
 $A(j,i,k) = A(i,j,k)$ is ok, but $A(k,j,i) = A(i,j,k)$ is not.
The work distribution between the nodes must be over the virtual index
rather than much smaller occupied index, to ensure that the algorithm
can scale into hundreds of nodes.  The work within the node can be
parallelized using threads.  This multi-level parallelization
guarantees that the algorithm will scale into thousands of cores.

The order in which algorithm points are listed are in the order in
which the algorithm was thought of.  Essentially, the primary focus is
on memory, then secondary storage and I/O, and only then on the
computational aspect.  It should be noted that despite that way of
thinking the performance does not suffer, as will be illustrated with
benchmarks.

\section{Implementation}
This section is broken down into three: the direct CCSD terms,
the non-direct CCSD terms, and triples correction.  The CCSD is by far
the most complex due to number of terms.

Before proceeding to the respective sections, a word must be said
about optimizing I/O via loop blocking.  Consider the Algorithm
 \ref{blocking}, where $B$ is a blocking factor. If $B = 1$, then
it is just a regular loop: the innermost (most expensive) load
operation is executed $N^3$ times, the total I/O overhead is $M^2N^3$,
and the local buffer size is $M^2$.  If $B$ is greater than 1, the
innermost load operation is called $(N/B)^3$ times, the I/O overhead
is $M^2B(N/B)^3 = M^2 N^3 /B^2$, and the local buffer size is $M^2B$.
So, at the cost of increasing local buffer, the I/O overhead can be
reduced by a factor of $B^2$.  In general, loop blocking decreases I/O
by $B^{(L-1)}$  where $L$ is the loop depth.

The loop blocking will be used where I/O might pose a problem.  Since
blocking also requires increase in memory overhead, the blocking
factor can be determined by setting runtime memory limit.

%\begin{lstlisting}[label=naive, caption=Naive approach, numbers=left]
\begin {listing}
\begin {verbatim}
for i = 0:N,B { // iterate to N in steps of B
  for j = 0:N,B {
    for k = 0:N,B {
      // the innermost load operation
      buffer(M,M,B) = load A(M,M,k:k+B)
      ...
    }
  }
}
\end{verbatim} \\
\hline
\caption{Loop Blocking}
\label{blocking}
\end{listing}


\subsection{Direct Terms}
As mentioned already, the $v^{ab}_{cd}$ has to be evaluated directly
due to storage constraints.  The same approach can be extended to
evaluate terms with $v^{ia}_{bc}$ directly as well at little
additional cost.

The $v^{ab}_{cd}$ is contracted with 
${T_1^2}^{cd}_{ij} = t^c_i t^d_j$ and
${T_2}^{cd}_{ij} = t^{cd}_{ij}$ amplitudes,

$${VT_2}^{ab}_{ij} = t_{ij}^{cd} C^s_d C^q_c V^{pr}_{qs} C^b_r C^a_p$$
$${VT_1^2}^{ab}_{ij} = t^c_i t^d_j C^s_d C^q_c V^{pr}_{qs} C^b_r C^a_p$$

Half-transforming the amplitudes to AO basis and factoring out
half-contracted terms yields expressions in terms of half-transformed
intermediates $U$,
$${U_2}^{qs}_{ij} = (t_{ij}^{cd} C^s_d C^q_c) V^{pr}_{qs}$$
$${U_1^2}^{qs}_{ij} = (t_{i}^{c} C^q_c) (t_j^d C^s_d) V^{pr}_{qs}$$
$${VT_2}^{ab}_{ij} = {U_2}^{qs}_{ij} C^b_s C^a_q$$
$${VT_1^2}^{ab}_{ij} = {U_1^2}^{qs}_{ij} C^b_s C^a_q$$

All similar $VT$ terms can be obtained from $U$ at virtually no cost by
having the last two AO indices transformed to occupied and virtual index.
For example, the $v^{ia}_{bc}$ terms in Eq. \ref{t1} is just
$$v^{ma}_{ef}(2 t^{ef}_{mi} - t^{ef}_{im}) =
2 U_{mi}^{qs} C_q^m C_s^a - U_{im}^{qs} C_q^m C_s^a$$

The $v^{ia}_{bc}$ also enters $VT_1$ diagrams,
$${VT_1}^{ij}_{ab} = t^c_i C^s_j C^q_c V^{pr}_{qs} C^b_r C^a_p$$
$${VT_1}^{ia}_{jb} = t^c_i C^s_d C^q_j V^{pr}_{qs} C^b_r C^a_p$$
and two more intermediates are needed,
$${U_1}^{ij}_{qs} = (t^i_a C^a_p) C^j_r V^{pr}_{qs}$$
$${U_1}^{ir}_{js} = (t^i_a C^a_p) C^q_j V^{pr}_{qs}$$
which can then be transformed into appropriate $VT_1$ diagrams.

Now, if all four $U$ intermediates are available, neither
$v^{ab}_{cd}$ nor $v^{ia}_{bc}$ need be stored for the CCSD
iterations, replaced with much smaller $4o^2n^2$ storage.

Half-transformed $T_2$ amplitudes also give a way to devise a direct
contraction algorithm with very little memory requirement.  Recall
that the contraction is in AO basis, and thus atomic indices can be
contracted without having to construct $V^{ab}_{qs}$ which would
require all $p,r$ indices and thus $N^2M^2$ memory, where $M$ is the
size of the largest shell.  The algorithm in Listing \ref{direct} only
needs $NM^3$ memory.

%\begin{lstlisting}[label=naive, caption=Naive approach, numbers=left]
\begin {listing}
\begin {verbatim}
for S in Shells {
  for Q <= S {
    for R in Shells {
      for P in Shells {
        // skip insignificant ints
        if (!screen(P,Q,R,S)) continue;
        G(P,R,Q,S) = eri(P,Q,R,S);
      }
      for r in R {
        U1(i,j,q,s) = ...
        U12(i,j,q,s) = ...
        load t(o,o,n,r)
        U2(i,j,q,s) += t(i,j,p,r)*G(p,r,q,s)
      }
    }
    store U1(i,j,Q,S), U1(j,i,S,Q)
    store U12(i,j,Q,S), U12(j,i,S,Q)
    store U2(i,j,Q,S), U2(j,i,S,Q)
  }
}
\end{verbatim} \\
\hline
\caption{Direct CCSD intermediates}
\label{direct}
\end{listing}

The important points of the algorithm \ref{direct} are:
\begin{itemize}
\item{The integral symmetry is exploited to halve integral calculation
     {\it and} transformations.}
\item{The loop over $Q,S$ can be distributed over nodes.}
\item{The loop over $R$ can be parallelized over threads.  In this
    case, the $U$ storage can be shared, provided the updates  to
    shared memory are synchronized.}
\item{The  innermost $t2$ loads can be reduced by bloking $Q,S$ loops.}
\item{Per thread storage is $NM^3$, which is 16MB for a basis set of
    size 2000 with $f$ shells ($M = 10$). The local $U$ storage is
    likewise small, only for 8MB $o=100$.  This tiny memory footprint
    allows for a very large $Q,S$ blocking factor and consequently the
    I/O can be drastically reduced.}
\end{itemize}

A careful reader may have noticed that both $UT_1$ terms can not be
evaluated simultaneously using the above algorithm, as they
corresponds to two different integrals, $\braket{pq}{rs}$ and
$\braket{pr}{qs}$.  However, one of them can be easily evaluated by
applying the algorithm a second time to compute a single $UT_1$ term
at a very modest $on^4$ cost.

\subsection{CCSD}
Because the singles amplitudes storage is negligible, $on$, the
singles is easy to implement and parallelize.  By making a virtual
index outermost, the local memory is guaranteed not to exceed $o^2n$
since all the diagrams with  three and four virtual indices have
already been evaluated above.

The doubles amplitudes calculation requires the most effort to
implement, primarily due to the number of contractions and the terms
which require significant I/O.  
Keep in mind that all $v^{ab}_{cd}$ and $v^{ia}_{bc}$ terms have been
evaluated, but so have been many similar $VT$ terms.

The first step towards deriving a scalable
algorithm for $Dt^{ij}_{ab}$ is to fix the outermost loop at $b$
index, which can be evaluated across nodes independently.
For each $b$ iteration a $o^2n$ $Dt$ block is evaluated and stored.

The quantities with $b$ index are loaded once,
guaranteed not to exceed $o^2n$ size.  The tensors without $b$
index imply that the tensor is needed in its entirety for each
index.  To ensure that no $v$ or $t$ memory exceeds $o^2n$, those
 tensors must be loaded into memory $o^2n$ tile at a time
for each $b$ index inside a loop over a dummy virtual index, lets call
it $u$.  This increases the I/O cost to $o^2n^3$, which is still below
the $o^3n^3$ computational cost.

There are three tensors which must be contracted fully for a given $b$
index: $v_{ia}^{jb}, v_{ij}^{ab}, t_{ij}^{ab}$.
The loop corresponding to $v_{ia}^{jb}$ can be elimiated
right away, it is only needed in its entirity to evaluate
$I_{ie}^{ma} t_{mj}^{eb}$ in Eq. \ref{t2}.  Since, this term appears
inside the symmetrizer $P$,
$$P(v_{je}^{mb} t_{mi}^{ea}) = P(v_{ie}^{ma} t_{mj}^{eb})$$
$I_{ie}^{ma}$ can be replaced by an equivalent $I_{je}^{mb}$.
This leads to Algorithm \ref{ccsd}.

%\itemize{lstlisting}[label=naive, caption=Naive approach, numbers=left]
\begin {listing}
\begin {verbatim}
for b in v {
  Dt(i,j,a) = 0

  load t(o,o,v,b)
  load V(o,o,v,b)
  load V(o,v,o,b)
  load V(o,o,o,b)

  Dt += Vt

  // terms with t
  for u in v {
    load t'(o,o,v,u)
    // evaluate terms with t'
    Dt += Vt'
  }

  // terms with v
  for u in v {
    load v'(o,o,v,u)
    // evaluate terms with v'
    Dt += V't
  }

  store Dt(o,o,v,b)
}
\end{verbatim} \\
\hline
\caption{CCSD Algorithm}
\label{ccsd}
\end{listing}

The important points about the Algorithm \ref{ccsd} are:
\begin{itemize}
\item The loop over $b$ index is easy to make parallel.
\item The local memory is on the order $4o^2n$ plus $o^2n$ per
  innermost $v'/t'$ memory.
\item The $b$ loop can be easily blocked to reduce the I/O by a
  blocking factor $B$ at the expense of increasing
  memory by a factor of $B$.
\item Since than memory footprint is low, B can be
  fairly large.   E.g. for O=100, V=2000, B=4 and B=8, the required memory is
  2.6G and 5.2G per {\it node} respectively.
\item  The operations outside the $u$ loop can be  parallelized inside
  the node by using threaded math library.
\item  The operations inside the $u$ loop can be explicitly
  parallelized inside the node via threads,
 with the added benefit of overlapping I/O and computations.
\end{itemize}


\subsection{(T)}
The $(T)$ correction, Eq. \ref{(t)}, only involves $t^{ij}_{ab}$,
$v^{ij}_{ka}$, $v^{ij}_{ab}$, and $v^{ia}_{bc}$.  The unused CCSD
arrays previously allocated can be freed to make space for
$v^{ia}_{bc}$.  Since $v^{ia}_{bc}$ was never constructed, another
integral transformation needs to be carried out at a small $on^4$
cost.

The original (T) correction equations were given in a way that
requires keeping occupied index fixed and permuting the virtual index,
in other words the local memory for $t^{ijk}_{abc}$ would have been $v^3$.
 Since triple amplitude are symmetric with respect to exchange of
 index ``columns'',
$$t^{ijk}_{abc} = t^{jik}_{bac} = t^{ikj}_{acb} = ...$$
all terms with $t^{jik}_{bac}$ can be written with virtual index
fixed, e.g.
$t^{ijk}_{bac} = t^{jik}_{abc}$,
$t^{ijk}_{cab} = t^{jki}_{abc}$, etc.

Now the $T_3$ amplitudes can be implemented as a series of 12 {\tt dgemms}
and 6 index permutations, Listing \ref{(t)}.
The important points about the algorithm are:
\begin{itemize}
\item The symmetry in $a,b,c$ indices is utilized.
\item The loop over $a,b,c$ indices is easily parallelizable.
\item Only the loads with $a$ index are innermost
\item The loops can be easily blocked to reduce the I/O by a factor of
  $B^2$ where B is the blocking factor.
\item The local storage required is $3o^2vB + 3o^3B + 6ovB^2 +
  o^3B^3$
\item If $B > 1$, the actual {\tt dgemms} are carried out inside
  another $B^3$ loop, which can be parallelized {\it within} a node by
  the means of   threads.
\item Since the memory footprint is low, blocking factor can be
  large.  E.g. for O=100, V=1000, B=4 and B=8, the required memory is
  1.6G and 6.4G per {\it node} respectively.
\end{itemize}

%\itemize{lstlisting}[label=naive, caption=Naive approach, numbers=left]
\begin {listing}
\begin {verbatim}
for c in V {
  for b in c {
    for a in b {

      load t(o,o,a,b)
      load t(o,o,a,c)
      load t(o,o,b,c)

      load v(o,o,o,a)
      load v(o,o,o,b)
      load v(o,o,o,c)

      load v(o,o,v,a)
      load v(o,o,v,b)
      load v(o,o,v,c)

      load v(o,v,b,c)
      load v(o,v,c,b)
      load v(o,v,a,c)
      load v(o,v,c,a)
      load v(o,v,a,b)
      load v(o,v,b,a)

      // t(i,j,e,a)*V(e,k,b,c) corresponds to dgemm(t(ij,e), V(e,k)), etc
      t(i,j,k) = t(i,j,e,a) V(e,k,b,c) - t(i,m,a,b) V(j,k,m,c)
      t(i,k,j) = t(i,k,e,a) V(e,j,c,b) - t(i,m,a,c) V(k,j,m,b)
      t(k,i,j) = t(k,i,e,c) V(e,j,a,b) - t(k,m,c,a) V(i,j,m,b)
      t(k,j,i) = t(k,j,e,c) V(e,i,b,a) - t(k,m,c,b) V(j,i,m,a)
      t(j,k,i) = t(j,k,e,b) V(e,i,a,c) - t(j,m,b,c) V(k,i,m,a)
      t(j,i,k) = t(j,i,e,b) V(e,k,c,a) - t(j,m,b,a) V(i,k,m,c)
      ...
    }
  }
}
\end{verbatim} \\
\hline
\caption{(T) Algorithm}
\label{(t)}
\end{listing}

\subsection{The overall picture.}
The algorithm is implemented entirely in C++, as a part of stand-alone
library which includes previous ERI, Fock, and MP2 methods
\cite{asadchev2010uncontracted,asadchev}.
The library requires only minimal input from the host program and can
be connected to a variety of packages.

The storage is implemented using Global Arrays (GA) \cite{ga} for
distributed memory and HDF5 \cite{hdf5} for file storage.  GAMESS's
own distributed memory interface (DDI) \cite{ddi} currently has no
support for arrays of more than 2 dimensions.  But with a small
addition, the DDI calls can be translated directly into GA
equivalents, so that the GAMESS can run via GA without modifications
while at the same time providing 3-d and 4-d array functionality via
direct calls to GA.

 The arrays are first allocated in faster GA memory until the limit is reached,
and then on the filesystem.  The arrays responsible for the most I/O
need to be allocated first to ensure they reside in distributed
memory.  The implementation is a separate library, linked to

Put together, the algorithm looks like this:
\begin{itemize}
\item{The CCSD arrays are allocated, with $t$ and $v_{ij}^{ab}$ first
  to ensure they are in the fast storage.  Overall, $t$,
  $v_{ij}^{ab}$, $v_{ij}^{ka}$, $v_{ia}^{jb}$, $v_{ij}^{kl}$, $Dt$,
  storage is needed.}
\item{The allocated arrays are evaluated using the regular 4-index
    transformation.}
\item{The initial $T2$ amplitudes are taken to be the MP2 amplitudes,
 $v_{ij}^{ab}/D_{ij}^{ab}$ and the $T1$ amplitudes are set to zero.}
\item{The intermediate $U$ storage is allocated.}
\item{The CCSD  equations are repeated until an acceptable threshold
    is reached, either the energy difference or the amplitude
    difference.}
\item{The CCSD step is optionally accelerated using DIIS
    \cite{scuseria1986accelerating}.}
\item{Once converged, all but the first three arrays are freed and
    $v_{ia}^{bc}$ array is allocated and and evaluated.}
\item{The non-iterative (T) method runs.}
\end{itemize}

\section{Performance}

To assess performance and applicability of the algorithm, the three
case scenarios are presented: single node performance,  departmental cluster
performance, and high-end cluster performance.  The inputs are
selected to reflect a range of basis functions and occupied orbitals.

The departamental cluster, Exalted, is composed of nodes connected by
InfiniBand Ethernet.  Each node has one Intel X5550 2.66GHz
6-core processor, 24GB of RAM, two local disk drives, and an NVIDIA
Fermi C2050 GPU card.

First, the capability of the algorithm to run on a single node and to
use filesystem in case not enough memory is available to store all
data.  The results four different inputs are in table \ref{node}.  As
can be seen, even on a single node fairly large jobs can still run in
a reasonable timeframe.  Despite falling back to disk in all the
cases, across the board the I/O time as a fraction of total time is
very small, below 5\%.

\begin{table}
  \label{node}
  \caption{Single Node Performance}
  \begin{center}
    \begin{tabular}{| l | c | c | c | c | c |}
      \hline
      Input                & \#AO/Occ & CCSD & (T) & (T) Mem/Disk &
      (T) I/O \\ 
      \hline
      C4N3H5/aug-ccPVTZ    & 565/21  & 42m   & 8h  & 2.1/19.5 GB & 13m \\
      C8H10N4O2/aug-ccPVDZ & 440/37  & 50m   & 17h & 5.5/17.0 GB & 13m \\
      SiH4B2H6/aug-ccPVQZ  & 875/16  & 141m  & 18h & 3.4/53.4 GB & 49m \\
      C8H10N4O2/ccPVTZ     & 640/37  & 180m  & 64h & 12.2/49.0 GB & 42m \\
      \hline
    \end{tabular}
  \end{center}
\end{table}

The cluster performance is assessed on the basis of the time larger 
jobs take to run,  Table \ref{cluster}, and the scalability of a
medium-size job, Table \ref{scaling}.
First, all of the inputs used for single node benchmarking can run in
 under a day.  Secondly, a large CCSD Tamoxifen calculation, C26H29NO,
 can run on this relatively small Exalted cluster, three hours per
 iteration.

As can be expected, the (T) algorithm scales well, Table
\ref{scaling}, since it is very easy to parallelize to a large number
of nodes. However, the scalability of the CCSD algorithm is not
perfect.  This is especially noticable when running on a large Cray
system,  Table \ref{cray}.  Nevertheless, the longer Tamoxfine calculation
scales reasonably well to 1024 cores.

 Each XE6 has to chips, 16 cores each.  The benchmarks in Table
 \ref{cray} were obtained running 32 threads over the entire node
 spawned from a single  MPI process.  the better option, especially in
 the case of (T)  is to run one MPI process per {\it core} rather than
 per {\it node}.  In the former case, the threads do not need to
 communicate over the slower bridge connecting two chips, Table
 \ref{cray2}.  Generally, there is a large penalty for sharing data
 across the {\it chips}, which  must be avoided by having flexible
 approach to launch jobs.

\begin{table}
  \label{cluster}
  \caption {Cluster Performance}
  \begin{center}
    \begin{tabular}{| l | c | c | c | c |}
      \hline
      Input                & \#AO/Occ & \# cores & CCSD & (T) \\ 
      \hline
      C4N3H5/aug-ccPVTZ    & 565/21  & 4         & 12   & 117 \\
      SiH4B2H6/aug-ccPVQZ  & 875/16  & 8         & 20   & 133 \\
      C8H10N4O2/ccPVTZ     & 640/37  & 8         & 26   & 482 \\
      C26H29NO/aug-ccPVQZ  & 961/71  & 16        & 211  & N/A \\
      \hline
    \end{tabular}
  \end{center}
\end{table}

\begin{table}
  \label{scaling}
  \caption{Cluster Scaling, C8H10N4O2/ccPVTZ}
  \begin{center}
    \begin{tabular}{| l | c | c |}
      \hline
      Cores/Nodes & CCSD & (T) \\ 
      \hline
      24/4  & 28 & 971 \\
      48/8  & 15 & 482 \\
      96/16 & 11 & 240 \\
      \hline
    \end{tabular}
  \end{center}
\end{table}

\begin{table}
  \label{cray}
  \caption {Cray XE6 Performance}
  \begin{center}
    \begin{tabular}{| l | c | c | c |}
      \hline
      \# cores                     & 256 & 512 & 1024 \\ 
      \hline
      SiH4B2H6 (T)/aug-ccPVQZ   & 130 & 76  & 42   \\
      C8H10N4O2 CCSD/ccPVTZ     & 15  & 9   & 6.5  \\
      C26H29NO CCSD/aug-ccPVQZ  & 253 & 134 & 76   \\
      \hline
    \end{tabular}
  \end{center}
\end{table}


\begin{table}
  \label{cray}
  \caption {Cray XE6 Intra-Node Configuration, SiH4B2H6 (T)/aug-ccPVQZ}
  \begin{center}
    \begin{tabular}{| l | c | c | c | c |}
      \hline
      \# cores & 32x1 Threads/MPI & 16x2 Threads/MPI \\ 
      \hline
      256      & 130 & 83 \\
      512      & 76 & 49  \\
      1024     & 42 & 27  \\
      \hline
    \end{tabular}
  \end{center}
\end{table}

\subsection{GPU}
As can be expected, the direct terms account for the most time in CCSD
iterations.  The favorable fact about implementation is that most of
that work is concentrated in continuous application of just one
 {\it dgemm} operation.  Adding a graphical processor (GPU) {\it dgemm} to
handle matrix multiplication, while keeping integral evaluation on the
host, is fairly easy.  However, in multithreaded environment, several
threads must be assigned to a GPU device to avoid work imbalance.

Augmented with GPU BLAS, via CUBLAS \cite{nath2011accelerating}, the
CCSD calculations get a noticeable speed up, Table \ref{gpu}, if the
direct term dominates the entire iteration (this is the case if number
of occupied orbitals is very small relative to basis set).  If the
number of occupied orbitals is a relatively high, the direct term
accounts for a smaller fraction of the total iteration time, and
consequently GPU benefit is less noticeable overall.

\begin{table}
  \label{gpu}
  \caption {Gpu CCSD performance, all times are in minutes per iteration}
  \begin{center}
    \begin{tabular}{| l | c | c | c|}
      \hline
      Input      & C8H10N4O2/ccPVTZ & SiH4B2H6/aug-ccPVQZ & C4N3H5/aug-ccPVTZ \\
      \hline
      Direct     & 124  & 131 & 36 \\
      Direct+GPU & 53   & 65  & 26 \\
      CCSD       & 163  & 142 & 42 \\
      CCSD+GPU   & 115  & 75  & 33 \\
      Speed-up   & 1.4x & 1.9X & 1.3X \\
      \hline
    \end{tabular}
  \end{center}
\end{table}

\section{Conclusions}
The algorithm  presented in this paper is able to handle fairly large
jobs on a single node, a small cluster, and high-end Cray  system.
The algorithm has reduced memory footprint and is able to optionally
use the filesystem if the data exceeds distributed memory storage.
The algorithm can also optionally use GPUs to speed up certain CCSD
computations.  When running on the multi-core node with multiple
processor packages (chips), algorithm benefits from limiting thread
communication to within a chip.

The algorithm is implemented entirely in C++, as a part of stand-alone
library which includes previous ERI, Fock, and MP2 methods.
\cite{asadchev}.  

\bibliographystyle{unsrt}% (uses file ``plain.bst'')
\bibliography{references}
\documentclass{beamer}
\usepackage[latin1]{inputenc}

\usepackage{graphicx}
\usepackage{listings}
\lstset{basicstyle=\tiny}

\newcommand{\bra}[1]{\langle #1|}
\newcommand{\ket}[1]{|#1\rangle}
\newcommand{\braket}[2]{\langle #1|#2\rangle}

\setlength{\parskip}{\baselineskip}

\usetheme{Warsaw}
\title[I/O and Memory]{I/O and Memory. \\ Experiences from Computation Chemistry}
\author{Andrey Asadchev}
\institute{VT.edu}

\begin{document}

\begin{frame}
  \titlepage
\end{frame}

\begin{frame}{Mountain Creek in Western Virginia}
\begin{figure}[here]
\begin{center}
\includegraphics[scale=0.07]{va.pdf}
\end{center}
\end{figure}
\end{frame}

\begin{frame}{Forewarning}
To keep the presentation short and accessible to a wide audience,
many theoretical details are glossed over and/or represented in simplified manner.
\end{frame}

\begin{frame}{A Very Brief Overview of Computation Chemistry}
\begin{itemize}
\item The goal is to obtain energies and properties of a molecule from {\em first principles}
  ie via the Schrodinger equation $H \Psi = E \Psi$
\item Most systems have no closed-form solution and must rely on approximations.
\item The wavefunction $\Psi$ in typically represented in terms of $N$ atomic basis functions $\chi$.
\item The wavefunction is then a linear combination, $\psi_i = \sum_b^{N} c_{ib} \chi_b$
\item Coefficients are optimized so as to minimize the energy.
\item The most common {\em first-order} method is Hartree-Fock (HF), which is an iterative method.
\item In its common formulation, $FC = EC$ HF requires computing so called two-electron integrals
  and diagonalizing the operator (eigenvalue problem)
\end{itemize}
\end{frame}

\begin{frame}{First I/O Bottleneck}
\begin{itemize}
\item The two-electron integrals do not change during the iterative HF steps.
\item They maybe precomputed and reused, as was done in the early days.
\item However, there are $N^4$ integrals, as opposed to $F$ operator which is $N^2$,
  and even with the modest basis of say 750 basis functions, the storage is 2.5TB
\item It was quickly realized that re-computing integrals on the fly was way more efficient.
\item This approach to HF is called direct HF.
\item In chemistry parlance, {\em direct} oftentimes refers to methods which recompute the intermediates.
\end{itemize}
\end{frame}

\begin{frame}{Past Hartree-Fock}
\begin{itemize}
\item The common wisdom is that HF about recovers 99\% of {\em total} energy and produces fairly accurate geometries.
\item The 1\% that Hartree-Fock leaves out is what is called the correlation energy.
\item The properties associated with that 1\% is what theorists usually seek.
\item The electron correlation methods try to recover that 1\% from the HF wavefunction.
\end{itemize}
\end{frame}

\begin{frame}{Electron correlation methods}
\begin{itemize}
\item Moller-Plesset Second Order (MP2) is a fairly cheap method, scales as $N^5$.
\item Coupled-Cluster (CC) methods are consider the best, but scale as $N^6$ and beyond.
\item Configuration Interaction (CI) method gives the exact result (within the limits of the basis set) but scales as $N!$
\item All electron correlation methods involve manipulations of large datasets, on the order of gigabytes and terabytes.
\item Over the years researcher developed methods to decrease storage and computation, for example resolution-of-identity (RI), density-fitting (DF), and other methods.  The cutting edge are the PNO methods (in my opinion).
\end{itemize}
\end{frame}

\begin{frame}{The Problem with Memory and I/O}
\begin{itemize}
\item The implementations typically try to store work data in memory,
  either on node or distributed across the cluster.
\item That limits the range of machines a user can utilize and pushes a finite resource (as opposed to infinite time).
\item Some data access must be fast, eg if the data is accessed in innermost loops or must be accessed in non-contiguous blocks.
\item Typically one would use various tricks such as pre-fetching, loop blocking, and computation/communication overlap to hide/decrease access latency.
\item But algorithms can be re-arranged and modified so that most of data is accessed only once per outer loop, in large blocks.
\item In that case, the data can be stored on the filesystem with minimal overhead.
\end{itemize}
\end{frame}

\begin{frame}{CI}
\begin{itemize}
\item CI is given as $\ket{\Psi} = c_0 \bra{\Phi_0} + \sum{c_i^a \bra{\Phi_i^a} } +  ..$
\item In essence, we want to consider every possible replacement from the reference.
\item In other words, we want to ``distribute'' $N$ electrons among $K$ orbitals:
 there are ${K \choose N_\alpha}$ alpha and ${K \choose N_\beta}$ beta arrangements (strings).
\item A particular configuration will be an alpha-beta string pair (determinant) and we want to consider ALL pairs.
\item The entire Hamiltonian is square the number of ALL pairs!!!
\item Only for smallest of values can the Hamiltonian be evaluated in full and diagonalized.
\item For anything larger, the CI roots (eigenvalues) and coefficient vectors (CI vectors)
  need to be evaluated iteratively (Davidson method)
\end{itemize}
\end{frame}

\begin{frame}{CI}
\begin{itemize}
\item The strings can be represented with 0s (empty orbitals) and 1s (electrons).
\item This leads to a bit strings representation, eg. $101 .. 100$.
\item For example $K = 4, N = 2$ configuration will generate $1100, 1010, 1001, 0110, 0101, 0011$
\item Bit strings can be indexed and manipulated efficiently using bit operators and hardware instruction: {\tt \&, >>, POPCNT, ...}.
\item The entire string list can sorted lexicographically and indexed using minimal perfect hash: \\
  {\tt hash(s) = Upper[16 >> s] + Lower[0xFFFF  \& s]}
\end{itemize}
\end{frame}


\begin{frame}{The three CI parts}
\begin{itemize}
\item The three types of CI excitations are alpha-alpha, beta-beta, and simultaneous alpha-beta.
\item Those are commonly referred to as $\sigma_1, \sigma_2, \sigma_3$. \\
\begin{align*}
\sigma_1(a,b) =& \sum_{b'} \sum_{ij}   { \bra{b} E_{ij}^{\beta} \ket{b'} h'_{ij} C(a,b') } + \\
& \frac{1}{2} \sum_{b'} \sum_{ijkl} { \bra{b} E_{ij}^{\beta} \ket{b'} \bra{b'} E_{kl}^{\beta} \ket{b''} (ij|kl) C(a,b') }
\\
\sigma_2(a,b) =&
  \sum_{a'} \sum_{ij}   { \bra{a} E_{ij}^{\alpha} \ket{a'} h'_{ij} C(a',b) } + \\
& \frac{1}{2} \sum_{a'} \sum_{ijkl} { \bra{a} E_{ij}^{\alpha} \ket{a'} \bra{a'} E_{kl}^{\alpha} \ket{a''} (ij|kl) C(a',b) }
 \\
\sigma_3(a,b) =& \frac{1}{2}\sum_{a'b'} \sum_{ijkl} {
  \bra{a} E_{ij}^{\alpha} \ket{a'} \bra{b} E_{kl}^{\beta} \ket{b'} (ij|kl) C(a',b')
}
\end{align*}
\end{itemize}
\end{frame}

\begin{frame}{The CI Dimensions}
 To give an idea of CI datasets
\begin{itemize}
\item (16 16) CI (16 orbitals, 8 alpha, 8 beta): 12,870 strings, 165 million determinants.  $C, \sigma$ each require 1.3GB
\item (20 20) CI (20 orbitals, 10 alpha, 10 beta): 184,756 strings, 34 billion determinants.  $C, \sigma$ each require 273GB
\item (21 20) CI (21 orbitals, 10 alpha, 10 beta): 352,716 strings, 124 billion determinants.  $C, \sigma$ each require 995GB.
\item Moreover, previous $C$ vectors must be retained in the Davidson expansion.
\end{itemize}
\end{frame}

\begin{frame}{Overall CI algorithm}
\begin{itemize}
\item CI algorithm can be broken down into two parts:  \\
  the calculation of $\sigma$ vector, $\sigma = \mathbf{H} c$ and Davidson iteration
\item $\sigma$ computation is computationally and I/O heavy, especially $\sigma_3$ part
\item The Davidson step is a straightforward linear algebra, primarily dot products and scaling.
\item The data for the Davidson step can be made local to each node.  There is no data transfer required other than the reduction of partial dot products. 
  Very straightforward to implement to utilize filesystem to the max.
\item Since the data will be local to a node, the I/O will scale with the number of nodes.
\end{itemize}
\end{frame}

\begin{frame}{Sigma 1/2}
\begin{itemize}
\item Sigma 2 computations can be formulated as
  $\sigma(a,b) = \sum_{b'} F(b \rightarrow b') C(a,b')$, where $F(b \rightarrow b')$
  depends only on beta string excitation and corresponding one- and two- electron integrals.
  Sigma 1 for $a'$ is defined similarly
\item In the closed-shell case, $C(a,b)$ is symmetric and $\sigma_1(a,b) = -1^{S} \sigma_2(b,a)$
\item The expression can be rewritten as $\sigma(b,a) = \sum_{b'} F(b \rightarrow b') C(b',a)$
\item In other words, to compute column $a$ of $\sigma$, we only need the corresponding $C$ column
\item Very easy to parallelize and to distribute data, \\
  distribute data column-wise and operate on local columns of $C$ and $\sigma$
\item In case of open-shell, transpose either matrix/vector between the two steps
\end{itemize}
\end{frame}

\begin{frame}[fragile]{Sigma 3}
\begin{itemize}
\item $\sigma_3(a,b) = \sum_{a'b'} F(a \rightarrow a', b \rightarrow b') C(a',b')$, 
\item Notice the big difference from earlier $\sigma$,
  neither rows nor columns ``match'' between $\sigma$ and $C$.
  That means a lot of I/O.
\item Lets say we need to compute a beta column $111000$.  Allowed excitations are  \\
$111000, 110100, 110010, 110001, 101100, ...$. \\
All those columns will need to be loaded.
\item What about a beta column $110100$?  Allowed excitations are \\
$111000, 110100, 110010, 110001, 101100, ...$. \\
That's a lot of overlap with a beta column $111000$.
\item I/O overhead can be hidden by computing a block of beta columns at a time.
\end{itemize}
\end{frame}

\begin{frame}[fragile]{Sigma 3}
 The general approach
\begin{itemize}
\item Pick a block of beta columns to compute.
\item For each of those beta strings, generate all valid excitations.
\item Sort excitation list
\item Loop over excitation list in blocks (list may have gaps)
\item Load $C(:,b')$ corresponding to that block
\item Evaluate $\sigma(:,b)$ columns corresponding to $C(:,b')$ block. \\
  Those $\sigma(:,b)$ columns aren't necessarily contiguous!
\item The I/O part is (kinda) taken care of.
\end{itemize}
\end{frame}

\begin{frame}[fragile]{Sigma 3}
  There is a lot of indexing involved in 
  $\sigma_3(a,b) = \sum_{a',b'} E_{ij}^{aa'} E_{kl}^{bb'} (ij|kl) C(a',b')$ \\
  We need to get valid excitations, phase factors, orbitals that are swapped, etc.
  One could try a few tricks.
\begin{itemize}
\item Make a subcopy of $(ij|kl)$, lets call it $(ij|b')$, corresponding to
  $b \rightarrow b'$ excitations in the block.
\item Apply $b \rightarrow b'$ phase factors to $(ij|b')$.
\item Now the $b'$ dimension no longer requires indirect indexing: \\
  $\sigma_3(a,b) = \sum_{a',b'} E_{ij}^{aa'} (ij|b') C(a',b')$.
\item Assuming column-major storage, need to transpose matrices.
\item There is no way to get rid of alpha indexing but
  for each alpha phase/index lookup there will be a block of vectorizable operations.
\end{itemize}
\end{frame}


\begin{frame}[fragile]{Performance}
Sample computation, 124,408,576,656 determinants, 64 nodes x 16 cores, 1024 total cores.
\begin{itemize}
\item Overall $\sigma$ time about 4300s, a bit over an hour.
\item $\sigma_{1+2}$ takes about 1300s
\item Out of that, $\sigma_{1+2}$ {\em kernel} accounts for about 50\%.
\item $\sigma_{1+2}$ I/O (read and write) is about 10\%.
\item $\sigma_{3}$ takes about 3000s, runs at about 20\% of max CPU rate.
\item Out of that, the {\em kernel} accounts for about two thirds of time.
\item I/O overhead isn't bad, on the order of 10\% but in-memory data reordering is rather expensive
\item With one-two expansion vectors, Davidson overhead is low, around 10\% of total wall time.
  Davidson wall time increases with the number of expansion vectors.
\end{itemize}
\end{frame}

\begin{frame}[fragile]{Further improvements}
Lexicographical ordering isn't the best.
\begin{itemize}
\item Consider the case of $\sigma_3$, for a rank $R$ string, only $R-1, R, R+1$ subset is needed.
\item With lexicographical ordering, the ranks are all over the place and I/O is not localized
\item Instead, (stable) sort the lexicographically ordered strings by their ranks.
\item Now the I/O will be limited only to neighbouring blocks $R-1, R, R+1$
\end{itemize}
\end{frame}

\begin{frame}[fragile]{Programming Aspects}
The object hierarchy is three-tiered
\begin{itemize}
\item In-core {\tt Tensor, Matrix, Vector}, derived from {\tt Eigen} library.
\item Distributed {\tt Array}, implemented in terms of various backends, eg ARMCI, HDF5, MPI, etc.
\item {\tt File} and file objects {\tt Dataset, Group, ...} implemented on top of HDF5.
\item Each object has {\tt range} access operators to give sub-block access.
\item In-core objects can be read/written to/from {\tt Array} and file objects.
\item {\tt Array} object can be read/written to/from file objects.
\end{itemize}
\end{frame}

\begin{frame}[fragile]{Example}
\begin{verbatim}
// create/open file
File file = File("file.h5");
// create dataset
File::Dataset<double> ds(file, "my dataset", dims);
Matrix a = Matrix::Random(m,n); // create some matrices
Matrix b = Matrix::Random(m-1,n-1);
ds << a; // write matrix to dataset
ds(range(0,m-1), range(0,n-1) >> b; // read subset
...
// create array, comm is MPI communicator
Array<double> array("my array", dims, comm);
a(0,2) = 3.14;
// write to distributed array
array(range(1,m), range(1,n)) << a;
\end{verbatim}
\end{frame}


\begin{frame}[fragile]{Questions.}
Questions.
\end{frame}


\end{document}


\end{document}



% LocalWords:  
