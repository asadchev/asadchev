\documentclass{beamer}
\usepackage[latin1]{inputenc}

\usepackage{graphicx}
\usepackage{listings}
\lstset{basicstyle=\tiny}

\newcommand{\bra}[1]{\langle #1|}
\newcommand{\ket}[1]{|#1\rangle}
\newcommand{\braket}[2]{\langle #1|#2\rangle}

\setlength{\parskip}{\baselineskip}

\usetheme{Warsaw}
\title[Group Meeting]{Final Defense Practice}
\author{ Andrey Asadchev}
\institute{VT.edu}

\begin{document}


\begin{frame}
  \titlepage
\end{frame}

\begin{frame}{About Me}
You know me already
\begin{figure}[here]
\begin{center}
\includegraphics[scale=0.5]{dog.pdf}
\end{center}
\end{figure}
\end{frame}



\begin{frame}{Previous Research}
\begin{itemize}
\item GAMESS (*vomit)
\item GPU (*vomit)
\item Rys Quadrature:  C++ with Mathematica optimized, Python generated expressions
\item Hartree-Fock
\item MP2:  can handle large systems, few thousand basis
\item CCSD(T):  runs on large and small machines, good I/O, disk as fallback
\end{itemize}
\end{frame}



\begin{frame}{Interests}
\begin{itemize}
\item C++
\item Python
\item Beautiful code and Domain specific languages
\item Hardware-level optimization and Parallel computing
\end{itemize}
CI - current project
\end{frame}


\begin{frame}{C++}
\item My evolution:  F77, F90, C, C++
\item General purpose language, aged well (C++11)
\item Templates and preprocessor
\item Domain specific language - eg Eigen
\item Access to hardware instructions
\item Not an easy language to master
\end{frame}


\begin{frame}{Python}
\item Very versatile, rapid prototyping language
\item Interacts with C/C++
\item Plethora of math and graphics packages
\item Mathematica-like shell for Quantum
\end{frame}


\begin{frame}{Beautiful code}
\item Code as expression of oneself, a bit artistic pursuit
\item Concise code lends itself to comprehension and easier manipulation
\item Easy manipulation leads to better algorithm
\item Domain specific languages (C++ or Python) allow expressing formulas concisely
\item Lower bar for beginners and more power to advanced folk.
\end{frame}


\begin{frame}{Optimization}
\item Few tricks: blocking, loop unrolling
\item Templates help a lot
\item Requires a little background in computer arch.
\item Parallel computing is usually data-driven in my approach (data storage determines the algorithm)
\item Some construct to transform loop to parallel would be nice (Rose)
\item In the end, {\it my} idea is to parallelize on algorithm level (rapid prototyping is handy)
\item Which leads to C++/Python alliance: C++ handles heavy lifting, Python ties the things together
\end{frame}


\begin{frame}{CI}
\item Large matrices (that need be stored for several cycles)
\item Some fun bit manipulation
\item Optimized to hide I/O (blocking)
\item Does ok atm, but needs more optimization
\item (16,16) Full CI in 6-40 mins (single thread)
\end{frame}

\end{document}
